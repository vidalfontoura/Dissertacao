\chapter{Conclusão}


In this work  GEHyPSP, an automatic way of generating high level heuristics to a hyper-heuristic framework for the PSP problem, was presented and evaluated. The PSP is a very challenging problem  with a high number of local optima and a very complex landscape. Many authors explored the PSP problem with heuristic methods. 
However, very often the proposed heuristic approaches are unable to find the best known results when executing against longer sequences. Usually, the hyper-heuristic framework fits well in this kind of complex scenario. Hence, the goal of this paper was to generate, using a grammatical evolution strategy, selection mechanisms and acceptance criteria to a hyper-heuristic framework and evaluate its performance and behavior with a set of eleven HP instances. Three groups of experiments were executed, using three randomly selected HP instances, in order to generate the high level heuristics and later the best individuals found in the experiments were executed using all the eleven HP instances. 

Three groups of experiments were executed: first generating only selection mechanisms within a fixed acceptance criterion; second generating only acceptance criteria using the best selection mechanism found in the first; finally both high level heuristics were generated in parallel. The results showed that better high level heuristics were found when generating them separately. Unfortunately, when analyzing the behavior of the generated high level heuristics against the eleven instances it was possible to see that they were not able to achieve the best known results for the longer sequences. However, when comparing with a good state-of-art human-designed hyper-heuristic framework (GIHH) \cite{misir2012intelligent} the results are slightly close. This fact shows that it is possible to automate the creation of high level heuristics and obtain results close to the state-of-art hyper-heuristics frameworks. 

Another finding of this work was the behavior of the best generated acceptance criteria. It was noticed that it behaves just like "a better or equal" human-designed move acceptance strategy. Also some of the generated acceptance criteria were always accepting worst solution and this fact impacted in the individual fitness. This fact demonstrates that GEHyPSP was able to generate acceptance criteria with the same behavior of simple human-designed move acceptance strategies. However, in order to obtain better results it might be necessary to improve GEHyPSP to generate more complex selection mechanisms and acceptance criteria to couple with the landscape complexity of the PSP problem. 

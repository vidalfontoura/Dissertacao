\chapter{Conclusão}
\label{cap:conclusao}


Nesta dissertação foram apresentadas duas abordagens heurísticas para o problema PDP utilizando o modelo simplificado HP-2D. 

A primeira abordagem consistiu em uma formulação multi objetiva para o PDP. Um novo objetivo foi adicionado o qual avalia o grau de compactação das estruturas de proteínas. Dessa maneira, considerando o valor da energia e o quão compacta uma dada estrutura é. AEMOs foram utilizados para explorar o espaço de busca no contexto multi objetivo. Os algoritmos utilizados foram o NSGAII \cite{deb2002} e o IBEA \cite{zitzler2004indicator}. Uma versão modificada desses algoritmos, com objetivo de obter melhores resultados, também foi proposta. Os algoritmos M\_NSGAII e o M\_IBEA são versões adaptadas utilizando uma inicialização via \textit{backtracking}  e um \textit{pool} de operadores. Portanto, 4 versões de AEMOs foram utilizadas para tentar resolver o PDP com o modelo HP-2D. Apenas a versão M\_IBEA conseguiu obter resultados expressivos em 7 instâncias de 11. Isto, demonstra que apenas a formulação multi objetiva não é suficiente para explorar o espaço de busca de maneira adequada.  


A segunda abordagem, denominada EGHyPDP, consistiu em gerar as heurísticas de alto nível para um \textit{framework} hiper heurístico, para o PDP utilizando o modelo HP-2D, através da evolução gramatical. Trés grupos de experimentos foram definidos o primeiro gerando apenas mecanismos de seleção; o segundo gerando apenas critérios de aceitação e o terceiro gerando ambos em paralelo. Analisando os resultados obtidos por cada grupo de experimento foi possível perceber que apenas o primeiro grupo de experimentos conseguiu obter resultados expressivos para 7 instâncias de 11. Os outros dois grupos de experimentos conseguiram obter bons resultados apenas para duas instâncias. Dessa maneira, demostrando que os critérios de aceitação gerados não são suficientes para guiar a busca para regiões promissoras. 

Infelizmente nenhuma das abordagens propostas nesta dissertação conseguiu obter os melhores resultados conhecidos para as maiores instâncias. Contudo, ambas as abordagens conseguiram obter bons resultados para 7 instâncias. Tanto o algoritmo M\_IBEA e a melhor heurística de alto nível gerada pelo EGHyPDP obtiveram os melhores resultados nas mesmas instâncias. Já nas instâncias mais complexas, nas quais não foi possível encontrar o melhor resultado conhecido, o M\_IBEA obteve resultados superiores ao EGHyPDP. Porém com diferença estatística segundo o teste de Kruskal-Wallis \cite{mckight2010kruskal} em apenas uma instância.

O PDP é um problema, em aberto,  extremamente desafiador princiapalmente por conta da enorme quantidade de possíveis existentes no espaço de busca. Além disso, existe uma grande variabilidade das características do espaço de busca para diferentes instâncias. Dessa maneira,  motivando muitos pesquisadores buscarem métodos mais elegantes e robustos para que estratégias heurísticas consigam explorar o espaço de busca de maneira apropriada.




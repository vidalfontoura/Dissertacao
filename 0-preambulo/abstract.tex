\begin{otherlanguage}{english}

\begin{abstract}

% em inglês, o primeiro parágrafo não deve ser indentado
\noindent

Proteins are structures composed by amino acids that plays a important role in nature. These structures are built by a process called protein folding, where a sequence of amino-acids initially unfolded will obtain your native structure. However, the protein folding process is not entire understood and it is considered one of the most challenging problem from biology, chemistry, medicine and bio-informatics. This problem is knows as the protein folding problem (PFP) and handles the prediction of protein structures.

The PFP is a minimization problem, because the proteins native structures are the one within minimum energy. Thus, many heuristics strategies make use of simplified models, such as the HP model, to find the proteins native structures within the HP model. Although simplified, the HP model has a complex search space and a great variability of characteristics between the instances. Thus, raises the demand of strategies with robust mechanisms to explore the search space properly. In this context, adaptive strategies fits well as good alternative to explore the fitness landscape from complex problems.
 
 In this dissertation, two approaches are presented to solve the PDP. The first one describes a bi objective approach applying traditional multi objective evolutionary algorithms. The second approach consists the automated design of high level heuristics using a genetic programming technique called grammatical evolution, which uses a grammar to produce computer programs.  

Both approaches proposed have been applied on a benchmark set with different amino acids sequences. The results have been compared with previous studies that used the same benchmark. In some cases the results obtained are promising which motivates the development of new adaptive strategies to solve the PFP.  


\end{abstract}

\end{otherlanguage}


\begin{otherlanguage}{english}

\begin{abstract}

% em inglês, o primeiro parágrafo não deve ser indentado
\noindent

Proteins are structures, composed by amino-acids, which plays a important role in nature. These structures are built by a process called protein folding, where a sequence of amino-acids initially unfolded will obtain your native structure. However, the protein folding process is not entire understood and it is considered one of the most challenging problem from biology, chemistry, medicine and bio-informatics. This problem is knows as the protein folding problem (PFP) and handles the prediction of protein structures. 

The PFP is a minimization problem, because the proteins native structures are the one within minimum energy. Thus, many heuristics strategies make use of simplified models, such as the HP model, to find the proteins native structures within the HP model. Since the PDP is a very complex problem with lots of local optima, it raises the necessity of building strategies that have robust mechanisms to maintain the diversity. In this context, that adaptive strategies fits well as good options to explore the fitness landscape from complex problems.

In this dissertation two approaches are presented for the PFP. The first one aims the application of multi objective evolutionary algorithms to solve the PFP. The second consisted on the automated design of high level heuristics through a type of genetic programming called grammatical evolution, which uses a BNF grammar to generate computer programs. 

Both strategies were applied using a benchmark set with different amino-acids sequences. The results were compared with other previous studies which used the same amino-acids sequences. In some cases the results obtained were promising that way motivating the development of new adaptive strategies to the PFP. 


\end{abstract}

\end{otherlanguage}


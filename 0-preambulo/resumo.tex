\begin{resumo}


Proteínas são estruturas, compostas por aminoácidos, que exercem um papel importante na natureza. Estas estruturas são formadas a partir de um processo de dobramento, no qual uma sequência de aminoácidos inicialmente desdobrada irá adotar uma conformação/estrutura espacial única/nativa. Entretanto, o processo de dobramento ainda não é completamente compreendido e é considerado um dos maiores desafios das áreas de biologia, química, medicina e bioinformática. Este desafio é conhecido como o problema de dobramento de proteínas (PDP) e trata da predição de estruturas de proteínas. 

O PDP pode ser visto como um problema de minimização, pois é afirmado que a estrutura nativa de uma proteína é aquela que minimiza sua energia global livre. Dessa maneira, diversas estudos aplicam estratégias heurísticas para explorar modelos simplificados. tais como o modelo HP. Embora simplificado, o modelo HP possui um complexo espaço de busca e uma grande variabilidade de características entre as instâncias. Dessa maneira surge a demanda de estratégias que possuam mecanismos robustos para explorar de maneira adequada o espaço de busca. É nesse contexto que hiper-heurísticas se apresentam como boas opções para explorar o espaço de busca de problemas complexos. 



Nesta dissertação, são apresentadas duas abordagens para resolver o PDP. A primeira descreve uma abordagem biobjetiva explorando  algoritmos evolucionários multi objetivos tradicionais. A segunda consiste no projeto automático de heurísticas de alto nível utilizando uma técnica de programação genética chamada evolução gramatical, a qual utiliza uma gramática para produzir programas de computador. 

 As estratégias propostas foram aplicadas utilizando um conjunto de \text{benchmark} com diferentes sequências de aminoácidos. Os resultados foram comparados com outros trabalhos que utilizam o mesmo conjunto de \textit{benchmark}. Alguns resultados obtidos se mostraram promissores dessa maneira motivando novos estudos que desenvolvam estratégias adaptativas para o PDP.



\end{resumo}


\begin{resumo}

O resumo deve conter no máximo 500 palavras, devendo ser 
Proteínas são estruturas, compostas por aminoácidos, que exercem um papel de suma importância na natureza. Estas estruturas são formadas a partir de um processo de dobramento, onde uma sequência de aminoácidos inicialmente desdobrada irá adotar uma conformação/estrutura espacial única/nativa. Entretanto, o processo de dobramento ainda não é completamente compreendido e é considerado um dos maiores desafios das áreas de biologia, química, medicina e bioinformática. Este desafio é conhecido como o problema de dobramento de proteínas (PDP) e trata da predição de estruturas de proteínas. Christian Anfinsen et al. descrevem que o PDP pode ser visto como um problema de minimização, pois afirmam que a estrutura nativa de uma proteína é aquela que minimiza sua energia global livre. Dessa maneira, diversas estratégias heurísticas utilizando modelos simplificados de representação de proteínas, tais como modelo HP, vem sendo desenvolvidas para buscar a estrutura nativa de proteínas. Por se tratar de um problema complexo com muitos mínimos locais e restrições, surge a demanda de estratégias que possuam mecanismos robustos para manter a diversidade assim como intensificar a busca quando necessário. É nesse contexto que hiper-heurísticas se apresentam como boas opções para explorar o espaço de busca de problemas complexos. 



Nesta dissertação, são apresentadas duas abordagens para o PDP. A primeira visa o uso de algoritmos evolucionários multi objetivos para resolver o PDP. A segunda consiste no \textit{design} automático de heurísticas de alto nível utilizando uma técnica de programação genética chamada evolução gramatical, a qual utiliza uma 

uma estratégia de geração de heurísticas de alto nível utilizando uma técnica de programação genética chamada evolução gramatical, a qual utiliza uma gramática BNF para gerar programas de computador. Um conjunto terminal e não terminal foi desenvolvido, baseado no trabalho de Sabar et al, e uma gramática específica para decodificar vetores de inteiros em heurísticas de alto nível. A evolução gramatical utiliza um processo evolutivo para evoluir uma população de programas. As heurísticas de alto nível irão compor uma plataforma hiper-heurística, a qual também possuirá um conjunto de heurísticas de baixo nível e um mecanismo de memória de soluções ao PDP. A estratégia proposta será aplicada utilizando um conjunto de \text{benchmark} com diferentes sequências de aminoácidos e comparada com outros trabalhos que utilizam o mesmo conjunto de \textit{benchmark}.




\end{resumo}


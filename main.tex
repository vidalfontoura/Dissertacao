% Este documento destina-se a servir como modelo para a produção de documentos
% de pesquisa do PPGINF/UFPR, como projetos, dissertações e teses. A classe de
% documento se chama "ppginf" (arquivo ppginf.cls) e define o formato básico do
% documento. O texto está organizado em capítulos que são colocados em
% subdiretórios separados. São definidos exemplos para a inclusão de figuras,
% códigos-fonte e a definição de tabelas.
%
% Produzido por Carlos Maziero (maziero@inf.ufpr.br) em Outubro de 2015.
% Adaptado de um modelo anterior construído pelo autor para o PPGIA/PUCPR.

% Opções da classe ppginf:
% - defesa: espaçamento 1,5, sem algumas páginas iniciais (default)
% - final:  espaçamento simples, completa
% - oneside: para impressão somente frente (default)
% - twoside: para impressão frente/verso
% - ... (demais opções aceitas pela classe "book")

% Opções default: defesa, oneside
\PassOptionsToPackage{table,xcdraw}{xcolor}
\documentclass[defesa,oneside]{ppginf}
%\documentclass[final,twoside]{ppginf}

% configurações de diversos pacotes, inclusive o fonte principal do texto
% Pacotes usados neste documento e suas respectivas configurações

% seleção de línguas do texto (a última é a principal/default)
\usepackage[english,brazilian]{babel}

% ------------------------------------------------------------------------------
% Definição de fontes

% formato dos arquivos-fonte (utf8 no Linux e latin1 no Windows)
\usepackage[utf8]{inputenc}	% arquivos LaTeX em Unicode (UTF8)

% usar codificação T1 para ter caracteres acentuados corretos no PDF
\usepackage[T1]{fontenc}

% fonte usada no corpo do texto (descomente apenas uma)
\usepackage{newtxtext,newtxmath}	% Times (se não tiver, use mathptmx)
%\usepackage{lmodern}			% Computer Modern (fonte clássico LaTeX)
%\usepackage{kpfonts}			% Kepler/Palatino (idem, use mathpazo)
%\renewcommand{\familydefault}{\sfdefault} % Arial/Helvética (leia abaixo)

% A biblioteca central da UFPR recomenda usar Arial, seguindo a recomendação da
% ABNT. Essa é uma escolha ruim, pois fontes sans-serif são geralmente inade-
% quados para textos longos e impressos, sendo melhores para páginas Web.
% http://www.webdesignerdepot.com/2013/03/serif-vs-sans-the-final-battle/.

% fontes usadas em ambientes específicos
\usepackage[scaled=0.9]{helvet}		% Sans Serif
\usepackage{courier}			% Verbatim, Listings, etc

% ------------------------------------------------------------------------------

% inclusão de figuras
\usepackage{graphicx}			% incluir figuras em PDF, PNG, PS, EPS

% subfiguras (subfigure is deprecated, don't use it)
\usepackage[labelformat=simple]{subcaption}
\renewcommand\thesubfigure{(\alph{subfigure})}

% ------------------------------------------------------------------------------



% inclusão/formatação de código-fonte (programas)
\usepackage{listings}
\lstset{language=c}
\lstset{basicstyle=\ttfamily\footnotesize,commentstyle=\textit,stringstyle=\ttfamily}
\lstset{showspaces=false,showtabs=false,showstringspaces=false}
\lstset{numbers=left,stepnumber=1,numberstyle=\tiny}
\lstset{columns=flexible,mathescape=true}
\lstset{frame=single}
\lstset{inputencoding=utf8,extendedchars=true}
\lstset{literate={á}{{\'a}}1  {ã}{{\~a}}1 {à}{{\`a}}1 {â}{{\^a}}1
                 {Á}{{\'A}}1  {Ã}{{\~A}}1 {À}{{\`A}}1 {Â}{{\^A}}1
                 {é}{{\'e}}1  {ê}{{\^e}}1 {É}{{\'E}}1  {Ê}{{\^E}}1
                 {í}{{\'\i}}1 {Í}{{\'I}}1
                 {ó}{{\'o}}1  {õ}{{\~o}}1 {ô}{{\^o}}1
                 {Ó}{{\'O}}1  {Õ}{{\~O}}1 {Ô}{{\^O}}1
                 {ú}{{\'u}}1  {Ú}{{\'U}}1
                 {ç}{{\c{c}}}1 {Ç}{{\c{C}}}1 }

% formatação de algoritmos
\usepackage{algpseudocode,algorithm,algorithmicx}
\floatname{algorithm}{Algoritmo}
\renewcommand{\algorithmiccomment}[1]{~~~// #1}
%\algsetup{linenosize=\footnotesize,linenodelimiter=.}

% ------------------------------------------------------------------------------

% outros pacotes
\usepackage{alltt,moreverb}	% mais comandos no modo verbatim
\usepackage{lipsum}		% gera texto aleatório (para os exemplos)
\usepackage{currfile}		% infos sobre o arquivo/diretório atual
\usepackage[final]{pdfpages}	% inclusão de páginas em PDF
\usepackage{longtable}		% tabelas multi-páginas (tab símbolos/acrônimos)

% listas de símbolos e de abreviações (a fazer)
%\usepackage[titles]{tocloft}
%\newlistof[part]{symb}{los}{Lista de Símbolos}
%\newlistof[part]{abbrev}{loa}{Lista de Abreviações}
%\newcommand{\symb}[2]{%
%\refstepcounter{symb}
%\addcontentsline{los}{symb}{\protect #1 :#2}\par}

\usepackage{syntax}
\grammarindent 80pt


\usepackage{newfloat}
% declare the floating environment {Grammar}
% this will also define \listofGrammars:
\DeclareFloatingEnvironment[
% the file extension for the file used to create the list:
fileext   = logr,% don't use log here!
% the heading for the list:
listname  = {List of Grammars},
% the name used in captions:
name      = Gramática,
% the default floating parameters if the environment is used
% without optional argument:
placement = htp
]{Grammar}








%=====================================================

\begin {document}

% Principais dados, usados para gerar as páginas iniciais.
% Campos não utilizados podem ser removidos ou comentados.

\title{Meta-Heurísticas e Hiper-Heurísticas  aplicadas ao problema de dobramento de proteínas}

% Estas devem ser definidas aqui para poder incorporar nos metadados do PDF
\pchave{PDP, hiper heurísticas, evolução gramatical, otimização multi objetiva}
\keyword{PFP, hyper heuristics, evolução grammatical, otimização multo objetiva}

\author{Vidal Daniel da Fontoura}
\advisor{Aurora Trinidad Ramirez Pozo}
\coadvisor{Roberto Santana}

\field{Ciência da Computação}		% default do PPGInf, não mudar

\date{2017}
\local{Curitiba PR}
\instit{UFPR}{Universidade Federal do Paraná}

%% Descrição do documento (obviamente, descomentar somente UMA!)

%\descr{Tese apresentada como requisito parcial à obtenção do grau de Doutor em Informática no Programa de Pós-Graduação em Informática, setor de Ciências Exatas, da Universidade Federal do Paraná}

%\descr{Documento apresentado como requisito parcial para o exame de qualificação de Doutorado no Programa de Pós-Graduação em Informática, setor de Ciências Exatas, da Universidade Federal do Paraná}

\descr{Dissertação apresentada como requisito parcial à obtenção do grau de Mestre em Informática no Programa de Pós-Graduação em Informática, setor de Ciências Exatas, da Universidade Federal do Paraná}

%\descr{Documento apresentado como requisito parcial para o exame de qualificação de Mestrado no Programa de Pós-Graduação em Informática, setor de Ciências Exatas, da Universidade Federal do Paraná}

%\descr{Trabalho apresentado como requisito parcial à conclusão do Curso de Bacharelado em XYZ, setor de Ciências Exatas, da Universidade Federal do Paraná}

%\descr{Trabalho apresentado como requisito parcial à conclusão da disciplina XYZ no Curso de Bacharelado em XYZ, setor de Ciências Exatas, da Universidade Federal do Paraná}

%=====================================================

% páginas iniciais (preâmbulo)
\frontmatter
\pagestyle{frontmatter}

% capa e folha de rosto
\titlepage

% páginas que só aparecem na versão final (inclusão automática)
% - IMPORTANTE - IMPORTANTE - IMPORTANTE - IMPORTANTE -
%
% O conteúdo exato da ficha catalográfica é preparada pela Biblioteca Central
% da UFPR, a pedido da secretaria do PPGINF. Não "invente" um conteúdo para ela,
% se informe a respeito com nossa secretária.

\begin{ficha}	% só gera conteúdo se for na versão final

% "Ficha" provisória (comentar quando tiver a ficha oficial)
\begin{center}
\textbf{Ficha Catalográfica}

~

\emph{Esta folha deve ser substituída pela ficha catalográfica fornecida pela Biblioteca Central da UFPR, a pedido da secretaria do PPGInf/UFPR (vide arquivo \texttt{ficha.tex}).}
\end{center}

% para a inclusão da ficha catalográfica final em formato PDF
%\includepdf[\thispagestyle{empty},noautoscale]{ficha.pdf}

\end{ficha}

%=====================================================
		% ficha catalográfica
% A ficha de aprovação será fornecida pela secretaria do programa,
% após a defesa e cumprimento dos demais trâmites legais.

\begin{aprovacao}	% só gera conteúdo se for na versão final

% "Texto" provisório (remover quando tiver a aprovacao oficial)
\begin{center}
\textbf{Termo de aprovação}

~

\emph{Esta folha deve ser substituída pela ata de defesa ou termo de aprovação devidamente assinado, que será fornecido pela secretaria do programa após a defesa ter sido concluída e aprovada (vide arquivo \texttt{aprovacao.tex}).}
\end{center}

% para a inclusão da aprovacao final em formato PDF:
%\includepdf[\thispagestyle{empty},noautoscale]{aprovacao.pdf}

\end{aprovacao}

%=====================================================
		% folha de aprovação
\include{0-preambulo/dedica}		% dedicatória
\include{0-preambulo/agradece}		% agradecimentos

% resumo e abstract
\begin{resumo}


Proteínas são estruturas, compostas por aminoácidos, que exercem um papel importante na natureza. Estas estruturas são formadas a partir de um processo de dobramento, no qual uma sequência de aminoácidos inicialmente desdobrada irá adotar uma conformação/estrutura espacial única/nativa. Entretanto, o processo de dobramento ainda não é completamente compreendido e é considerado um dos maiores desafios das áreas de biologia, química, medicina e bioinformática. Este desafio é conhecido como o problema de dobramento de proteínas (PDP) e trata da predição de estruturas de proteínas. 

<<<<<<< HEAD
O PDP pode ser visto como um problema de minimização, pois é afirmado que a estrutura nativa de uma proteína é aquela que minimiza sua energia global livre. Dessa maneira, diversas estudos aplicam estratégias heurísticas para explorar modelos simplificados, tais como o modelo Hidrofóbico-Polar HP. Embora simplificado,  HP possui um complexo espaço de busca e uma grande variabilidade de características entre as instâncias. Por conta de tal complexidade surge a demanda de estratégias que possuam mecanismos robustos para explorar de maneira adequada o espaço de busca. É nesse contexto que hiper-heurísticas se apresentam como boas opções para explorar o espaço de busca de problemas complexos. 
=======
O PDP pode ser visto como um problema de minimização, pois é afirmado que a estrutura nativa de uma proteína é aquela que minimiza sua energia global livre. Dessa maneira, diversas estudos aplicam estratégias heurísticas para explorar modelos simplificados. tais como o modelo HP. Embora simplificado, o modelo HP possui um complexo espaço de busca e uma grande variabilidade de características entre as instâncias. Dessa maneira surge a demanda de estratégias que possuam mecanismos robustos para explorar de maneira adequada o espaço de busca. É nesse contexto que hiper-heurísticas se apresentam como boas opções para explorar o espaço de busca de problemas complexos. 
>>>>>>> 1491eafdc569b7a670d8f8b83aae192c363e6aab



Nesta dissertação, são apresentadas duas abordagens para resolver o PDP. A primeira descreve uma abordagem biobjetiva explorando  algoritmos evolucionários multi objetivos tradicionais. A segunda consiste no projeto automático de heurísticas de alto nível utilizando uma técnica de programação genética chamada evolução gramatical, a qual utiliza uma gramática para produzir programas de computador. 

<<<<<<< HEAD
 As estratégias propostas foram aplicadas sobre um conjunto de \text{benchmark} com diferentes sequências de aminoácidos. Os resultados foram comparados com outros trabalhos que utilizam o mesmo conjunto de \textit{benchmark}. Alguns resultados obtidos se mostraram promissores dessa maneira motivando novos estudos que desenvolvam estratégias adaptativas para o PDP.
=======
 As estratégias propostas foram aplicadas utilizando um conjunto de \text{benchmark} com diferentes sequências de aminoácidos. Os resultados foram comparados com outros trabalhos que utilizam o mesmo conjunto de \textit{benchmark}. Alguns resultados obtidos se mostraram promissores dessa maneira motivando novos estudos que desenvolvam estratégias adaptativas para o PDP.
>>>>>>> 1491eafdc569b7a670d8f8b83aae192c363e6aab



\end{resumo}


\begin{otherlanguage}{english}

\begin{abstract}

% em inglês, o primeiro parágrafo não deve ser indentado
\noindent

Proteins are structures composed by amino acids that plays a important role in nature. These structures are built by a process called protein folding, where a sequence of amino-acids initially unfolded will obtain your native structure. However, the protein folding process is not entire understood and it is considered one of the most challenging problem from biology, chemistry, medicine and bio-informatics. This problem is knows as the protein folding problem (PFP) and handles the prediction of protein structures.

The PFP is a minimization problem, because the proteins native structures are the one within minimum energy. Thus, many heuristics strategies make use of simplified models, such as the HP model, to find the proteins native structures within the HP model. Although simplified, the HP model has a complex search space and a great variability of characteristics between the instances. Thus, raises the demand of strategies with robust mechanisms to explore the search space properly. In this context, that adaptive strategies fits well as good options to explore the fitness landscape from complex problems.
 
 In this dissertation, two approaches are presented to solve the PDP. First one describes a bi objective approach applying traditional multi objective evolutionary algorithms. The second approach consists the automated design of high level heuristics using a genetic programming technique called grammatical evolution, which uses a grammar to produce computer programs.  

Both approaches proposed were applied using an benchmark set with different amino acids sequences. The results were compared with previous studies that used the same set. In some cases the results obtained were promising that way motivating the development of new adaptive strategies to the PFP.  


\end{abstract}

\end{otherlanguage}



% sumário e demais listas (figuras, tabelas, abreviações/siglas, símbolos)
\tableofcontents
\listoffigures
\listoftables
%=====================================================

% lista de acrônimos (siglas e abreviações)

\begin{listaacron}

\begin{longtable}{ll}
DINF & Departamento de Informática\\
PPGINF & Programa de Pós-Graduação em Informática\\
UFPR & Universidade Federal do Paraná\\
\end{longtable}

\end{listaacron}

%=====================================================
		% ainda deve ser preenchida à mão
%=====================================================

% lista de símbolos

\begin{listasimb}

\begin{longtable}{ll}
$RC$ & \textit{Reward Credit}, crédito obtido.\\
$C_{best}$ & \textit{Current Best Solution}, melhor solução atual.\\
$C_{current}$ & \textit{Current Solution}, solução atual\\
$C_{accept}$ & \textit{Current Solution Accpted}, vezes que a solução atual foi aceita.\\
$C_{ava}$ & \textit{Average Reward Credit}, média de créditos obtidos anteriormente. \\
$C_r$ & \textit{Times Ranked First} número de vezes classificada como primeiras.\\

$Delta$ & \textit{Fitness Difference}, a diferença de qualidade\\
 $PF$ & \textit{Previous Fitness}, a qualidade da solução anterior.\\
 $CF$ & \textit{Current Fitness}, a qualidade da solução atual.\\
 
 $CI$ & \textit{Current Iteraction}, iteração corrente.\\
 $TI$ & \textit{Total Number of Iteractions}, número total de iterações.

\end{longtable}

\end{listasimb}

%=====================================================
		% idem

%=====================================================

% corpo do documento
\mainmatter
\pagestyle{mainmatter}

% inclusao de cada capítulo, alterar a gosto (do professor de Metodologia...)
\chapter{Introdução}

As proteínas são responsáveis por muitas funções importantes das células vivas. Estas estruturas garantem o correto funcionamento de um amplo número de entidade biológicas. As proteínas são o produto de um processo chamado de dobramento de proteínas, no qual uma cadeia de aminoácidos inicialmente desdobrada será transformada em sua estrutura final/nativa. A predição de estruturas de proteínas possui um campo amplo de aplicações biotecnológicas e médicas. Por exemplo: síntese de novas proteínas e dobramentos \cite{wang2012structural, rothlisberger2008kemp}, síntese de novas drogas baseada nas estruturas \cite{qian2004improvement, krieger2009improving}  e obtenção experimental de estruturas a partir de dados incompletos de ressonância magnética nuclear \cite{shen2009novo}.  

Determinar a estrutura nativa de proteínas uma tarefa desafiadora até mesmo para modernos super computadores. Isto ocorre por conta do imenso espaço de busca para testar todas as possíveis configurações que uma dada proteína pode adotar. Diferentes formas de representar as estruturas/conformações de proteínas existem e podem ser utilizadas para simular o processo de dobramento. Embora existam modelos extremamente detalhados, estas representações são computacionalmente muito custosas. Consequentemente, muitos autores \cite{custodio2004investigation,hsu2003growth,lin2011protein,unger1993genetic,santanna2008,custodio2014multiple, garza2012locality} utilizam modelos simplificados para representar as estruturas de proteínas. Um modelo bastante conhecido para este propósito, criado por Lau and Dill \cite{lau1989lattice}, é o modelo Hidrofóbico-Polar (HP). Este modelo simplifica os amino ácidos em apenas dois tipos: hidrofóbico (H) e polar (P). 



Também existe uma grande variação das características do espaço de busca entre diferentes instâncias do modelo HP. \textit{Grids} bidimensionais (2D) e tridimensionais (3D) pode ser utilizados para representar as conformações que uma proteína pode adotar. Cada conformação no \textit{grid} tem um valor de energia associado.

Muitos estudos  \cite{custodio2004investigation,hsu2003growth,lin2011protein,unger1993genetic,santana2008component,custodio2014multiple, garza2012locality} have used the Hydrophobic-Polar (HP) \cite{lau1989lattice}
 partir do processo de dobramento. Para avaliar as estruturas representadas pelo modelo HP é preciso computar o valor de energia associado a uma dada conformação \cite{unger1993genetic}. Para isto, é necessário considerar as interações entre os aminoácidos. Uma interação ocorre quando o par de aminoácidos é adjacente no \textit{grid}/cubo e não é adjacente na sequência. No modelo HP existem apenas três: (HP, HH e PP), mas somente interações hidrofóbicas (HH) influenciam no valor de energia referente a uma conformação \cite{unger1993genetic}. A questão que surge é como buscar, dentre as possíveis conformações, aquela cuja a energia seja mínima.
			% introdução
\include{2-pdp/pdp}
\chapter{Referencial Teórico}
\label{cap:Referencial Teórico}

% figuras estão no subdiretório "figuras/" dentro deste capítulo
\graphicspath{\currfiledir/figuras/}

Este capítulo apresenta o referencial teórico das estratégias heurísticas implementadas para resolver o problema PDP. Inicialmente, os algoritmos evolutivos e otimização multi objetiva são apresentados. Em seguida os algoritmos multi objetivos: NSGAII e IBEA são descritos em detalhe. Na sequência hiper heurísticas são apresentadas. O capítulo termina descrevendo programação genética, evolução gramatical e a abordagem básica para aplicar estas técnicas para produzir programas de computador.


\section{AEs - Algoritmos Evolucionários}
\label{sec:aemos}


Um algoritmo evolucionário (AE) é uma técnica de busca, altamente paralela, inspirada na teoria da seleção natural e reprodução genética de Charles Darwin. De acordo com a teoria de Darwin, a seleção natural irá favorecer os indíviduos que forem mais aptos, dessa maneira, estes indíviduos tem uma maior probabilidade
de reprodução. Indivíduos com mais descendentes tem uma chance maior de perpetuarem seus códigos genéticos nas gerações futuras. O código genético é a identidade de cada indivíduo e é representado por cromossomos. Estes princípios são utilizados na implementação de algoritmos computacionais, que procuram por soluções melhores para um dado problema, evoluindo uma população 
de soluções codificadas em cromossomos artificais -- estruturas de dados utilizadas para representar soluções factíveis para um dado problema \cite{pacheco1999algoritmos}.

De maneira geral, problemas reais de otimização  estão presentes em muitas áreas do conhecimento epossuem múltiplos objetivos a serem minimizados/maximizados. Para otimizar problemas multiobjetivos dois ou mais objetivos são considerados, os quais geralmente são conflitantes. Para estes problemas é impossível encontrar uma
única solução ótima. Um conjunto de soluções é encontrado avaliando a dominância de Pareto \cite{pareto} entre as soluções. O objetivo principal é encontrar o conjunto  
de soluções que sejam não dominadas entre si. Uma solução domina outra, se e somente se, for melhor em pelos um dos objetivos, sem ser pior em qualquer outro.
Este conjunto de soluções constitui a fronteira de Pareto. Encontrar a fronteira real de Pareto é um problema NP-Completo \cite{fonseca2005tutorial}, dessa maneira,
o objetivo é encontrar uma boa aproximação da fronteira.

Algoritmos Evolucionários Multi-Objetivos (AEMOs) são extensões de AEs para problemas multi-objetivos, os quais aplicam
conceitos da dominância de Pareto para criar diferentes estratégias para evoluir e manter a diversidade das soluções.
Nesta dissertação foram explorados dois AEMOs: NSGAII \cite{deb2002fast} e IBEA \cite{zitzler2004indicator}. Estes algoritmos foram selecionados pois sãp tradicionais e muito utilizados em diversos problemas do mundo real.
%TODO: TALVEZ REFERENCIAR
%=====================================================

\subsection{NSGAII - Non-dominated sorting Genetic Algorithm II}
\label{subsection:nsgaii}

O algoritmo \ref{alg:nsgaII} apresenta o pseudo código do NSGAII. O algoritmo recebe como parâmetro $N$ o tamanho da população e $T$ o número máximo 
de avaliações. O algoritmo inicia criando uma população com tamanho $N$ chamada $P_0$. A população $P_0$ é classificada de acordo com aptidão 
e a relação de não dominância. A população $P_0$ é submitida ao operador de seleção: torneio binário para selecionar duas soluções que serão utilizadas
para gerar descendentes. Operadores de cruzamento e mutação são aplicados na soluções selecionadas gerando duas soluções distintas descendentes. 
Ao fim do processo de reprodução, as soluções descendentes são avaliadas e adicionadas a população chamada $Q_0$.

Após esta estapa, $P_0$ e $Q_0$ são adicionadas em uma população auxiliar chamada $R$. Utilizando o conceito de não dominância, $R$ é ordenada 
criando fronteiras, onde cada solução da primeira fronteira não é dominada por nenhuma outra solução, já soluções da segunda fronteira são dominadas
apenas por soluções contidas na primeira fronteira, e assim por diante. Para cada fronteira, as soluções são avaliadas utilizando um mecanismo
de \textit{Crowding-Distance (CD)}. Soluções com maiores valores de CD irão ser adicionadas na população da próxima geração chamada $P_t$, no qual $t$ é a
avaliação corrente.

Após criar e preencher $P_t$ com as soluções não dominadas de todas as fronteiras, a população $P_t$ é avaliada e então passa para um novo
torneio binário e reprodução, dessa maneira, iniciando uma novo ciclo do algoritmo.

\begin{algorithm}[htb!]
	\begin{algorithmic}[1]
		\State{$N \gets$ Population Size}
		\State{$T \gets$ Max evaluations}
		\State{$P_0 \gets CreatePopulation(N);$}
		\State{$CalculateFitness(P_0);$}
		\State{$FastNonDominatedSort(P_0);$}
		\State{$Q_0 \gets 0$}
		\While{$Q_0 < N$}
		\State{$Parents \gets BinaryTournament(P_0);$}
		\State{$Offspring \gets CrossoverMutation(Parents);$}
		\State{$Q_0 \gets Offspring$}
		\EndWhile
		\State{$CalculateFitness(Q_0);$}
		\State{$t \gets 0$}
		\While{$t < T$}
		\State{$R_t \gets P_t \cup Q_t;$}
		\State{$Fronts \gets FastNonDominatedSort(R_t);$}
		\State{$P_{t+1} \gets 0$}
		\State{$i \gets 0$}
		\While{$P_{t+1} + Front_i  < N$}
		\State{$CrowdingDistanceAssignment(Front_i);$}
		\State{$P_{t+1} \gets P_{t+1} \cup Front_i$}
		\State{$i \gets i + 1$}
		\EndWhile
		\State{$CrowdingDistanceSort(Front_i);$}
		\State{$P_{t+1} \gets P_{t+1} \cup Front_i[1:(N -P_{t+1})]$}
		
		\State{$Parents \gets BinaryTournament(P_{t+1});$}
		\State{$Q_{t+1} \gets CrossoverMutation(Parents);$}
		\State{$t \gets t +1$}
		\EndWhile
		\State{\Return{$P \gets$ Set of non-dominated solutions.}}
	\end{algorithmic}
	\caption{NSGAII}
	\label{alg:nsgaII}
\end{algorithm}


%=====================================================

\subsection{IBEA (Indicator-Based Evolutionary Algorithm)}

No contexto de otimização multiobjetiva, otimizar consiste em tentar encontrar a fronteira com uma boa aproximação da fronteira real de Pareto. 
Entretanto, não existe uma definição geral para "uma boa aproximação". Consequentemente, indicadores de qualidade vem sendo utilizados
para avaliar a qualidade da aproximação de fronteiras. O indicador \textit{hypervolume} é um exemplo de indicador utilizado para avaliação e comparação 
das fronteiras.

No algoritmo IBEA, indicadores de qualidade são	utilizados para avaliar o conjunto de soluções não dominadas \cite{figueiredo2013algoritmo}.
Para utilizar o IBEA, é necessário definir qual indicador será utilizado para associar cada solução a um valor scalar. Um dos indicadores bastante utilizados é o \textit{hypervolume} por conta da sua capacidade de avaliar a convergência e a diversidade do espaço de busca ao mesmo \cite{ishibuchi2008evolutionary}.

\begin{equation} \label{eq:ibeaFitness}
F(x_i) = \sum_{x_j \in (P-x_i)} {-e^\frac{-I_{Hy}(x_j,x_i)}{k}}
\end{equation}

A equação de \textit{fitness} do IBEA é apresentada pela equação \ref{eq:ibeaFitness} e é utilizada para calcular a contribuição de uma dada solução
para o valor do indicador referente a população, onde $k$ é um fator escalar dependente do $I_{Hy}$, o qual representa o indicador de qualidade sendo utilizado. O valor $F(x_i)$ corresponde à perda de qualidade da aproximação, da fronteira real de Pareto, se a solução $x_i$ for removida da população \cite{figueiredo2013algoritmo}.


O Algoritmo \ref{alg:ibea} recebe como parametro o tamanho da população $N$, o número máximo de avaliações $T$ e o fator escalar $k$. O processo se inicia
criando uma população $P$ de tamanho $N$. Os seguintes passos irão se repetir até que o critério de parada seja atingido: um torneio binário 
para selecionar indivíduos, reprodução (cruzamento e mutação) dos indivíduos selecionados para gerar descendentes, adicionar os descedentes na população
auxiliar $\overline P$. Após a reprodução, a população $\overline P$ é unida com $P$. Enquanto o tamanho de $P$ exceder $N$, o pior indíviduo avaliado
pela equação \ref{eq:ibeaFitness} é removido da população $P$ e os indíviduos restantes são re-avaliados. Quando o algoritmo terminar irá retornar o conjunto
de soluções não dominadas é retornado.

\begin{algorithm}[htb!]
	\begin{algorithmic}[1]
		\State{$N \gets$ Population Size}
		\State{$\overline N \gets$ AuxiliaryPopulationSize}
		\State{$T \gets$ Max Evaluations}
		\State{$k \gets$ Scale factor of Fitness}
		
		\State{$P \gets$ CreatePopulation($N$);}
		\State{$\overline P \gets$ CreateEmptyAuxiliaryPopulation($\overline N$);}
		\State{$m \gets 0$}
		\State{CalculateFitness($P$);}
		
		\While{$m \ge T$ or other stop criterion is not reached}
		
		\State{$\overline P \gets$ BinaryTournament($P$);}
		\State{$\overline P \gets$ CrossoverMutation($\overline P$);}
		\State{$P \gets P \cup \overline P$}
		\State{$m \gets m+1$}
		
		\While{Size($P$) $> N$}
		\State{$x^* \gets$ WorstIndividualByFitness();}
		\State{RemoveFromPopulation($x^*$, $P$);}
		\State{CalculateFitness($P$);}
		\EndWhile
		
		\EndWhile
		\State{\Return{$P \gets$ Set of non-dominated solutions}}
		
	\end{algorithmic}
	\caption{IBEA}
	\label{alg:ibea}
\end{algorithm}

%\footnotetext[1]{\textit{Hypervolume}: Indicador de qualidade utilizado neste estudo \cite{zitzler1998multiobjective}, 
%	denotado como o "tamanho da área coberta do espaço de busca". Este indicador tem uma vantagen importante em relação aos outros \cite{zitzler2007hypervolume}:
%	1 - Sensitivo a qualquer tipo de melhoria na aproximação em relação a outro conjunto. 

\section{Hiper-Heurísticas }
\label{Hiper-Heuristicas}

Apesar do sucesso de métodos heurísticos e outros métodos de busca na tarefa de resolver problemas de busca computacional difíceis, ainda existem dificuldades em generalizar estes métodos para diferentes problemas ou até mesmo para diferentes instâncias de um mesmo problema. Esta dificuldade provém principalmente da necessidade de selecionar os parâmetros e configurações mais adequados dos algoritmos para um problema ou para uma dada instância de um problema. Também vale mencionar a pouca orientação na tarefa de definir estes parâmetros.  É neste contexto que surge uma questão: é possível automatizar o projeto e parametrização de métodos heurísticos para resolver problemas de busca computacional difíceis? \cite{burke2013hyper}. A ideia principal é desenvolver algoritmos que sejam mais genéricos do que as implementações de metodologias atuais \cite{burke2013hyper}. As principais abordagens já propostas para este desafio podem ser classificadas em duas categorias: configuração estática (\textit{offline}) e controle dinâmico (\textit{online}). Abaixo são apresentadas algumas abordagens já propostas na literatura:

\begin{itemize}
	\item Configuração Estática (Offline):
	\begin{itemize}
		\item Seleção de algoritmos;
		\item Portfólio de algoritmos;	
		\item Configuração de algoritmos;
		\item Ajuste de parâmetros;
		\item Hiper-Heurísticas.
	\end{itemize}
	\item Controle Dinâmico (Online):
	\begin{itemize}
		\item Seleção adaptativa de operadores (AOS);
		\item Controle de parâmetros;	
		\item Algoritmos meméticos adaptativos;
		\item Hiper-Heurísticas
	\end{itemize}
	
\end{itemize}


Esta seção tratará apenas de hiper-heurísticas e suas particularidades. Uma hiper-heurística pode ser vista como uma metodologia de alto nível, a qual seleciona ou cria heurísticas para resolver um dado problema ou instância de um problema. \cite{burke2013hyper}. O objetivo principal é tentar encontrar ou construir a heurística mais adequada para cada situação. As hiper-heurísticas diferem dos métodos padrão de busca, pois operam sobre o espaço de busca de heurísticas que por sua vez operam sobre o espaço de busca de um problema. Além disso, hiper-heurísticas são independentes do problema. Tradicionalmente \textit{frameworks} hiper-heurísticos possuem dois níveis \cite{sabar2015automatic}: 

\textbf{Heurísticas de baixo nível}:  Um conjunto de heurísticas de baixo nível específicas. Estas heurísticas costumam diferir entre domínios de problemas. São exemplos: operadores de cruzamento, mutação e buscas locais. Em alguns casos, meta-heurísticas, dependendo da modelagem do \textit{framework} hiper-heurístico, também podem assumir o papel de heurísticas de baixo nível. 

\textbf{Heurísticas de alto nível}: Geralmente consistem em dois componentes: mecanismo de seleção, que gerencia quais heurísticas de baixo nível devem ser aplicadas durante a busca; um critério de aceitação, que tem a responsabilidade de decidir se irá aceitar ou não uma solução gerada, a partir da aplicação de uma heurística de baixo nível. A responsabilidade do mecanismo de seleção é selecionar, de um conjunto de heurísticas de baixo nível, a heurística que for mais adequada naquele momento. Geralmente, a escolha da heurística de baixo nível é crucial para uma boa exploração do espaço de busca, evitando que a busca fique confinada em uma região específica \cite{sabar2015automatic}. O objetivo do critério de aceitação é auxiliar o processo de busca a evitar mínimos locais assim como explorar diferentes regiões através do aceite ou rejeição de soluções geradas \cite{chakhlevitch2008hyperheuristics}. Espera-se que um bom critério de aceitação deva atingir um bom equilíbrio entre aceitar soluções melhores assim, como soluções diversificadas caso a busca esteja presa em um mínimo local \cite{sabar2015automatic}. Ambos os componentes devem ser independentes de conhecimento sobre o problema.

A Imagem \ref{img:hiperheuristico} apresenta um diagrama exemplificando os níveis de um \textit{framework} hiper-heurístico e suas características. Note que entre os níveis (alto e baixo) existe uma barreira de domínio, ou seja, apenas as heurísticas de baixo nível são dependentes de conhecimento do problema ou instância enquanto as heurísticas de alto nível não são dependentes do problema. 

\begin{figure}[!htb]
	\centering
	\includegraphics{Imagens/HiperHeuristicas.png}
	\caption{Framework Geral Hiper-Heurístico. Adaptado de \cite{sabar2015automatic}}
	\label{img:hiperheuristico}
\end{figure}


Como cada instância ou problema possui um espaço de busca com diferentes características, os componentes da heurística de alto nível têm um grande impacto no desempenho de um \textit{framework} hiper-heurístico. Esta é uma das razões de existir um grande interesse de pesquisa em desenvolver  novos mecanismos de seleção, assim como diferentes critérios de aceitação \cite{burke2013hyper}. Um bom mecanismo de seleção deve selecionar a heurística mais adequada em um dado momento, para guiar a busca para regiões promissoras do espaço de busca. 
Ao utilizar hiper-heurísticas, espera-se encontrar o método correto ou a sequência de heurísticas que mais se adequam a um problema ou instância em vez de tentar resolver o problema diretamente. Entretanto, um importante objetivo é desenvolver métodos genéricos, que têm  potencial em produzir soluções com uma qualidade aceitável, utilizando um conjunto de heurísticas de baixo nível com fácil implementação. As hiper-heurísticas podem ser classificadas de diversas maneiras. A Figura \ref{img:classificacaoHiperHeuristicas} apresenta as possíveis classificações descritas na literatura. 

\begin{figure}[!htb]
	\centering
	\includegraphics[scale=0.8]{Imagens/ClassificacaoHiperHeuristica.png}
	\caption{Classificação Hiper-heurísticas. Adaptado de \cite{sabar2015automatic}}
	\label{img:classificacaoHiperHeuristicas}
\end{figure}

A primeira classificação de hiper-heurísticas é baseada na sua fonte de conhecimento durante a busca: \textit{Online} é quando a hiper-heurística toma decisões de maneira instantânea, baseando-se em métricas durante sua execução, não necessitando de treinamento prévio. \textit{Offline} necessita de treinamento prévio; estes \textit{frameworks}  tomam suas decisões baseados no que foi aprendido apenas durante o treinamento, sem atualização deste conhecimento. Os \textit{frameworks} classificados como \textit{No-Learning} não possuem nenhuma forma de aprendizagem. Outra classificação considera a forma como as heurísticas de baixo nível operam sobre as soluções do problema. As heurísticas ditas perturbativas realizam pequenas perturbações nas soluções gerando novas soluções. Já heurísticas construtivas criam soluções do zero passo a passo e normalmente avaliam cada etapa da construção para obter \textit{feedback} sobre o seu desempenho. Uma última  classificação, mas não menos importante, divide as hiper-heurísticas de acordo com a  natureza do seu espaço de busca. As hiper-heurísticas de seleção selecionam sequências de heurísticas a serem aplicadas para resolver um dado problema ou instância. Já as hiper-heurísticas de geração operam gerando novas heurísticas com objetivo de resolver um problema ou instância.


\subsection{Hiper-heurísticas de Geração}
\label{Hiper-Heuristicas-Geraçao}

Estas hiper-heurísticas geram novas heurísticas combinando componentes de heurísticas existentes \cite{burke2013hyper}. Geralmente se utiliza programação genética (GP), como hiper-heurística para gerar heurísticas. A próxima sub-seção irá introduzir o conhecimento necessário para a compreensão da programação genética, assim como irá introduzir evolução gramatical, que se trata de um tipo de programação genética e que será utilizada nesta proposta.

\section{Programação Genética (PG)}
\label{subsection:PG}

Programação Genética \cite{burke2009exploring} é um ramo da síntese de programas que utiliza ideias oriundas da teoria da evolução natural para produzir programas. Da mesma maneira, que nos AEs um processo evolutivo é aplicado a uma população de indivíduos. Na PG um indivíduo representa um programa de computador. Os operadores geneticamente inspirados (cruzamento e mutação) são repetidamente aplicados com objetivo de produzir novos programas de computador. Estes programas são avaliados utilizando uma função de \textit{fitness} (normalmente dependente do desempenho obtido pela aplicação do programa em um problema), que determina quais destes programas são mais suscetíveis a sobreviver para gerações futuras. Os programas com maior aptidão tem mais chances de serem selecionados para o cruzamento e perpetuarem parte de seus códigos genéticos durante o processo evolutivo. 
Programação genética é um método de geração de programas sintaticamente válidos e a função de \textit{fitness} é utilizada para decidir quais programas são mais adequados para o problema.
Na programação genética, os programas que compõem a população são tradicionalmente representados utilizando estruturas de árvore. Existem outras estruturas que podem ser evoluídas, por exemplo: sequências lineares de instruções ou gramáticas. Nesta  Dissertação será utilizada uma representação gramatical linear que será explicada na seção \ref{subsubsection:EvolucaoGramatical}.

\section{Evolução Gramatical (EG)}
\label{subsubsection:EvolucaoGramatical}

Evolução gramatical é uma técnica relativamente nova de computação evolutiva, proposta por Ryan et al. \cite{ryan1998grammatical}. Trata-se de um tipo de programação genética. Assim como na programação genética, o principal objetivo é encontrar um programa executável ou trecho de um programa, que obtenha um bom valor de \textit{fitness} para o problema em questão. Na maioria dos trabalhos publicados de programação genética, expressões que representam estruturas de árvore são manipuladas, enquanto na evolução gramatical os operadores genéticos são aplicados em vetores de inteiros que posteriormente são mapeados para um programa (ou trecho de programa) através de uma gramática específica. Um dos benefícios de EG é que este mapeamento generaliza a aplicação para diferentes linguagens de programação. \cite{ryan1998grammatical} propõem uma técnica para gerar programas ou fragmentos de programas para qualquer linguagem de programação utilizando notações \textit{Backus Naur Form} (BNF). Esta técnica pode ser utilizada para evoluir programas por um processo evolutivo. A evolução gramatical adota um mecanismo de mapeamento entre o genótipo (indivíduos codificados em um vetor de inteiros) e o fenótipo (programas gerados para resolver algum problema). 
A notação BNF é utilizada para expressar a gramática de uma linguagem na forma de regras de produção. Uma gramática BNF consiste em um conjunto de terminais, os quais são itens que podem aparecer na linguagem, por exemplo: +, -, *, / etc e não terminais, que podem ser expandidos em um ou mais terminais e não terminais. Uma gramática pode ser expressada como uma tupla ${N,T,P,S}$, onde $N$ é o conjunto de não terminais, $T$ o conjunto de terminais, P um conjunto de regras de produção que mapeia os elementos $N$ para $T$; e, por último, $S$, um símbolo de início e que está contido em $N$.
A Gramatica \ref{gram:gramatica} apresenta um exemplo de definições BNF para gerar funções aritméticas simples.

\begin{center}
	
	$ N = {\langle expr \rangle, \langle op \rangle, \langle pre-op \rangle}$
	
	$ T = {Sin,Cos,Tan,Log,+,-,/,*,X} $
	
	$ S = \langle expr \rangle $
	
\end{center}

\noindent
E $P$ pode ser representada como:

\begin{Grammar}
	\begin{grammar}
		
		
		<expr> ::=  <expr> <op> <expr> \hspace{10cm} (0) 
		\alt (<expr> <op> <expr>) \hspace{9.7cm} (1)  
		\alt <pre-op> (<expr>) \hspace{10.15cm} (2) \alt <var> \hspace{12.1cm} (3) \\\
		
		<op> ::=  + \hspace{12.7cm} (0)   \alt - \hspace{12.8cm} (1)  \alt  /  \hspace{12.85cm} (2) \alt * \hspace{12.75cm} (3) \\
		
		<pre-op> ::= Sin  \hspace{12.4cm} (0) \alt Cos
		\hspace{12.3cm} (1) \alt Tan  \hspace{12.35cm} (2)
		
		<var> ::= X  \hspace{12.6cm} (0)
		
		
	\end{grammar}
	
	\caption{Gramática exemplo para demonstrar como decodificar vetores de inteiros em programas de computador.}
	\label{gram:gramatica}
\end{Grammar}


\begin{table}[htb]
	\centering
	\caption{\textit{Regras de produção} e o número de escolhas para cada uma.}
	\label{tab:productionRules}
	\begin{tabular}{|l|l|}
		\hline
		Regra de produção & Número de escolhas \\ \hline
		$\langle expr \rangle$                        & 4       \\ \hline
		$\langle op \rangle$                         & 4       \\ \hline
		$\langle pre-op \rangle$                         & 3       \\ \hline
		$\langle var \rangle$                          & 1       \\ \hline
	\end{tabular}
\end{table}


Ryan et al. \cite{ryan1998grammatical}  propôs o uso de um algoritmo genético (AG) para controlar quais escolhas devem ser feitas, permitindo dessa maneira que o AG controle quais regras de produção serão utilizadas. Um indivíduo (cromossomo) consiste em um vetor de tamanho variável de valores inteiros que representa o genótipo. Para fins de compreensão o processo de mapeamento de um cromossomo será demonstrado utilizando a \autoref{gram:gramatica}. O Algoritmo \ref{alg:pseudocodigogrammar} apresenta o \textit{template} geral dos programas gerados pela \autoref{gram:gramatica}. A expressão $\langle expr \rangle$ apresentada na linha 2 do Algoritmo \ref{alg:pseudocodigogrammar} é substituída por expressões matemáticas que estão codificadas pelos cromossomos (vetores de inteiros). 

\begin{algorithm}
	\caption{\textit{Template} geral dos algoritmos gerados}
	\label{alg:pseudocodigogrammar}
	float symb(float x) { \\
		a = $\langle expr \rangle$;   \\
		return a;  \\
	}	
\end{algorithm}

\noindent
Suponha o seguinte vetor de inteiros:

\begin{center}
	$ [220, 203, 17, 6, 108, 215, 104, 30] $
\end{center}


Este vetor será utilizado para mapear o cromossomo (genótipo) em um trecho de programa (fenótipo) utilizando a gramática BNF. 
%A expressão não terminal $ \langle expr \rangle$ no algoritmo \ref{alg:pseudocodigogrammar} será preenchida por um trecho de código que será mapeado a partir do cromossomo apresentado. Os passos do mapeamento serão descritos a seguir.%
A tabela \autoref{tab:productionRules} resume o número de escolhas associada à cada regra de produção da \autoref{gram:gramatica}. Existem 4 opções de regras de produção que podem ser selecionadas para a expressão $ \langle expr \rangle$. Para decidir qual será selecionada, o primeiro valor no cromossomo deve ser utilizado. Sendo que o valor é 220. Devemos realizar o módulo deste valor pelo número de escolhas, neste caso 4. Portanto, 220 MOD 4 = 0, o que significa selecionar a primeira opção: $\langle expr \rangle \langle op \rangle \langle expr \rangle$.

Note que a primeira expressão é novamente $ \langle expr \rangle$ e da mesma maneira devemos obter o próximo valor de inteiro e realizar o módulo. O próximo valor inteiro é 203; realizando o modulo de 4, resulta em 3, que portanto seleciona a quarta opção: $ \langle var \rangle$. Substituindo na expressão anterior, obtemos: $ \langle var \rangle \langle op \rangle \langle expr \rangle$

Nenhuma escolha é necessária para a expressão $ \langle var \rangle$, pois existe apenas uma opção $X$. A expressão pode ser reescrita da seguinte maneira: $X \langle op \rangle \langle expr \rangle$

Neste momento é necessário decodificar a expressão não terminal $\langle op \rangle$. Obtendo o próximo valor inteiro do cromossomo, temos 17 e para o $ \langle op \rangle$ temos 4 opções $(+ | - | / | *)$. O resultado de 17 MOD 4  é igual a 1, que significa selecionar:  $-$. Substituindo na expressão, temos: $X  -  \langle expr \rangle$


Novamente é necessário fazer uma nova escolha para resolver a expressão não terminal $\langle expr \rangle$. O próximo valor do cromossomo é 6 e novamente existem 4 opções. Realizando o modulo 6 MOD 4, obtém-se 2, que seleciona $ \langle pre-op \rangle ( \langle expr \rangle)$. Atualizando a expressão, obtemos: $X - \langle pre-op \rangle (\langle expr \rangle)$

Resolvendo a expressão $ \langle pre-op \rangle$, obtemos 108 MOD 4 = 0 que por sua vez seleciona a primeira expressão  terminal $Sin$. Atualizando a expressão, obtemos: $X - Sin (\langle expr \rangle)$

Expandindo $ \langle expr \rangle$, obtemos 215 MOD 4 = 3, que seleciona a expressão não terminal $ \langle var \rangle$. Já que para a expressão $ \langle var \rangle$ existe apenas uma opção, nenhuma escolha é necessária e a expressão final decodificada (fenótipo) é: $X - Sin (X)$

Note que nem todos os genes do cromossomo foram necessários para obter o fenótipo. Nos casos em que isto ocorre, os genes que não forem utilizados são desconsiderados. Além disso, pode ocorrer que um cromossomo não tenha genes suficientes para mapear um programa. Neste caso a estratégia é reutilizar os genes do cromossomo a partir do primeiro gene. 

Operadores genéticos tradicionais (cruzamento e mutação) também são utilizados na EG. Além dos operadores tradicionais outros dois operadores \textit{Prune} e \textit{Duplicate} são peculiares à EG e serão descritos em seguida:

\begin{itemize}
	\item \textit{Duplicate}: Este operador (dada uma probabilidade) realiza a cópia de  alguns genes. Os genes duplicados são adicionados após a última posição do cromossomo. O número de genes a serem duplicados é selecionado de maneira aleatória. A motivação por trás deste operador é que ao duplicar genes ocorre um aumento da presença de genes que são potencialmente bons, pois pertencem a um indivíduo com boa aptidão selecionado pelo operador de seleção.
	\item \textit{Prune} : Este operador leva em consideração que nem sempre todos os genes, de um cromossomo, são utilizados para decodificar um programa. Dessa maneira (dada uma probabilidade) realiza o truncamento de  cromossomos. O objetivo é diminuir a probabilidade que o operador de cruzamento opere em regiões dos cromossomos que não sejam utilizadas realmente.
\end{itemize}


O Algoritmo \ref{alg:GE} apresenta o pseudocódigo da evolução gramatical (EG). Note que o pseudocódigo é muito similar a um algoritmo genético simples. Nas linhas 3 e 4 ocorre a inicialização da população e o mapeamento para programas utilizando a gramática que foi provida como entrada. Em seguida, na linha 5 ocorre a execução dos programas e na linha 6 acontece a avaliação dos indivíduos da população, baseando-se na saída obtida pelos respectivos programas. Dentro do laço principal, apresentado na linha 7, podemos observar o processo de seleção dos indivíduos pais na linha 8 e na linha 9 o processo de cruzamento destes indivíduos. Nas linhas 10 e 11 ocorre a aplicação dos operadores \textit{Prune} e \textit{Duplicate} respectivamente e na linha 12 podemos observar a aplicação do operador de mutação. Em seguida, nas linhas 13,14 e 15 ocorre o mapeamento dos indivíduos descendentes para programas, execução dos programas e finalmente a atribuição de \textit{fitness} para os descendentes. Por fim, na linha 16 do laço principal, ocorre a substituição dos descendentes na população. 


%exceto pela aplicação dos operadores \textit{Duplicate} e \textit{Prune} (linhas 10 e 11 do Algoritmo  \ref{alg:GE}) e o processo de decodificação e execução dos programas descendentes (linhas 13 e 14 do Algoritmo \ref{alg:GE}).

\begin{algorithm}[htb!]
	%\fontsize{8pt}{10pt}\selectfont
	
	
	\begin{algorithmic}[1]
		\State{$AG  \gets$ Arquivo da gramática;}
		\State{$população \gets$ Inicialização a população;}
		\State{$programas \gets$ Mapeia $população$ para programas utilizando $AG$;}
		\State {Executa os $programas$;}
		\State {Atribui valor de \textit{fitness} para as soluções  of $população$ de acordo com a saída obtida pelos respectivos programas decodificados;}
%		\While{Condição de parada não atingida}
		\While{Condição de parada não atingida}
			\State {$pais \gets $ Seleção de indivíduos para cruzamento;}
			\State {$descendentes \gets$ Cruzamento $pais$;}
			\State {Aplica o operator \text{Prune} nas soluções $descendentes$;}
			\State {Aplica o operador \textit{Duplicate} nas soluções $descendentes$;}
			\State {Aplica o operador de mutação nas soluções $descendentes$;}
			\State {$programas \gets$ Mapeia $descendentes$ para programas utilizando $AG$;}
			\State {Executa $programas$;}
			\State {Atribui valor \textit{fitness} para as soluções $descendentes$ de acordo com a saída obtida pelos respectivos programas decodificados;}
			\State 	{$população \gets$ Realiza substituição;}
		\EndWhile \\
		\Return{Melhor programa da $população$;}
			
			
		
	\end{algorithmic}
	\caption{Pseudocódigo da evolução gramatical}
	\label{alg:GE}
\end{algorithm}


\section{Programação Genética como Hiper-Heurística de Geração de Heurísticas}
\label{subsubsection:PGasHH}

Nesta seção serão apresentadas questões relativas ao uso de EG como mecanismo de geração de heurísticas. Burke et al. \cite{burke2009exploring} descrevem que muitos autores mencionam a melhor adequação de programação genética, em relação a outras técnicas de aprendizagem de máquina, para gerar heurísticas de maneira automática. Burke et al. também apontam algumas vantagens desta técnica:

\begin{itemize}
	\item PG utiliza cromossomos de tamanho variável. Geralmente, não se sabe um tamanho ótimo para representar heurísticas de um dado domínio de problema.
	\item PG produz estruturas de dados executáveis. E heurísticas são tipicamente expressadas como programas ou algoritmos.
	\item Facilidade em identificar boas características do domínio do problema, afim de definir o conjunto terminal que será utilizado pela PG.
	\item Heurísticas desenvolvidas por humanos podem facilmente ser expressadas na mesma linguagem utilizada para criar o espaço de busca da PG. O conjunto de funções, relevante para o problema pode ser determinado facilmente. E adicionalmente PG pode ser suplementada com uma gramática específica.
\end{itemize}

Todas estas vantagens descritas por \cite{burke2009exploring} também são consideradas ao utilizar EG, visto que se trata de uma extensão de programação genética e possui as mesmas características (cromossomo de tamanho variável, produz estruturas executáveis, etc). \cite{burke2009exploring} também mencionam desvantagens, por exemplo: a cada execução da programação genética é encontrada uma melhor heurística que, por se tratar de uma técnica estocástica, os resultados podem ser distintos em diferentes execuções. Portanto, se fazem necessárias múltiplas execuções a fim de se obter um melhor conhecimento da qualidade das heurísticas que podem ser produzidas. Outra desvantagem é referente à configuração de parâmetros, que normalmente é encontrada via tentativa e erro.

\subsubsection{Abordagem Básica}

\cite{burke2009exploring} descrevem uma abordagem básica para aplicar programação genética para gerar heurísticas:

\begin{enumerate}
	\item Examinar as heurísticas existentes: Avaliar se as heurísticas já propostas para um dado problema podem ser descritas em um \textit{framework} comum. Estas heurísticas podem ter sido criadas por humanos ou até mesmo concebidas via outras técnicas de aprendizagem. Este passo não é trivial, pois envolve o entendimento de um número diverso de heurísticas existentes, que podem operar de diferentes maneiras. Geralmente heurísticas desenvolvidas por humanos são produtos de anos de pesquisa, e portanto, uma boa compreensão das heurísticas existentes pode ser um trabalho difícil. 
	\item Um framework que utilizará as heurísticas: neste momento a preocupação é em como as heurísticas serão aplicadas para um dado problema. Em geral, os frameworks tendem a ser bem diferentes dependendo do domínio do problema. 
	\item Definição do conjunto terminal: neste passo a preocupação refere-se a variáveis que expressem o estado do problema. Estas variáveis irão compor os terminais da programação genética/evolução gramatical. Outros terminais também podem ser utilizados. Particularmente, constantes aleatórias podem ser úteis.
	\item Definição do conjunto de funções: é necessário definir como as variáveis estarão relacionadas ou combinadas entre si. Estes relacionamentos irão compor o conjunto de funções da programação genética/evolução gramatical. 
	\item Identificar uma função de \textit{fitness}: uma função de \textit{fitness} precisa ser identificada para o problema. Geralmente, uma função simples de aptidão não irá avaliar bem os cromossomos. Introduzir alguns parâmetros pode ajudar a encontrar uma mais adequada.
	\item Executar o framework: geralmente ao executar pela primeira vez um framework hiper-heurístico com programação genética, não serão produzidos bons resultados, devido à escolha dos parâmetros. Isto é observado especialmente em casos que o pesquisador é iniciante. Portanto, é essencial que as definições de parâmetros sejam cuidadosamente investigadas.
\end{enumerate}




\section{Considerações Finais}
\label{ReferencialTeorico:Conclusão}


Neste capítulo foram apresentados alguns conceitos que permeiam a área de estudo otimização multi objetiva e sobre hiper-heurísticas. Inicialmente, uma contextualização sobre AEs foi fornecida em seguida os AEMOs foram introduzidos. Nesta dissertação foram utilizados os algoritmos NSGAII e IBEA os quais também foram apresentados. Também foi discutido sobre \textit{frameworks} hiper-heurísticos,  os seus níveis (alto e baixo) e as classificações encontradas na literatura.  Foram discutidas algumas estratégias para hiper-heurísticas de seleção e geração. As hiper-heurísticas de geração foram mais detalhadas, pois esta proposta visa o projeto  automático de heurísticas de alto nível. Foram apresentados os conceitos de PG e sua extensão EG, por se tratarem de estratégias comumente utilizadas para o projeto de hiper-heurísticas de geração de heurísticas. Também foram discutidas algumas vantagens e desvantagens referentes ao uso de PG para geração de heurísticas, além de demonstrar que a EG possui as mesmas características da PG, pois se trata de uma extensão que utiliza uma gramática para gerar os programas. O funcionamento geral da EG foi demonstrado utilizando uma gramática exemplo e um vetor de inteiros e, por fim, o pseudocódigo da evolução gramatical foi apresentado. 

O capítulo \ref{cap:Trabalhos Relacionados} apresenta os trabalhos relacionados que foram selecionados a partir da revisão bibliográfica realizada.



		% fundamentação teórica
\chapter{Trabalhos Relacionados}
\label{cap:Trabalhos Relacionados}

Este Capítulo irá apresentar alguns trabalhados relacionados com a presente Dissertação. Serão apresentados trabalhos que buscam construir/adaptar estratégicas heurísticas para encontrar melhores soluções ao PDP utilizando o modelo HP. Também serão apresentados alguns trabalhos que utilizam técnicas de programação genética para gerar heurísticas para diferentes problemas.


%Também será apresentado um trabalho que trata do \textit{design} automático de heurísticas de alto nível para um \textit{framework}  hiper-heurístico aplicado a problemas de \textit{benchmark} disponibilizados pelo \textit{software} HyFlex \cite{ochoa2012hyflex}.

%O estudo, desenvolvido por \cite{unger1993genetic}, foi percussores, ao aplicar um algoritmo genético ao PDP com o modelo HP, utilizando operadores de cruzamento e mutação aprimorados. Os resultados apresentados superam um número significativo de estratégias tradicionais anteriores que utilizam métodos Monte Carlo para explorar as conformações.



Um algoritmo genético multi memético foi proposto em \cite{krasnogor2002multimeme}. Esta estratégia combina um algoritmo genético e buscas locais selecionando a busca local que mais se adequar com a instância (sequência) sendo otimizada. Mais tarde este trabalho foi aprimorado com um estratégia \text{fuzzy} para as buscas locais, dessa maneira produzindo melhores resultados para o PDP.

Em \cite{hsu2003growth}, os autores utilizam um algoritmo de crescimento de cadeia, chamado \textit{pruned-enriched Rosenbluth method} (PERM). Esta estratégia se baseia em iterativamente construir uma conformação adicionando os aminoácidos um a um. 

A otimização de colônia de formigas também foi aplicada para o PDP nos trabalhos \cite{shmygelska2002ant,shmygelska2003improved}. Estas abordagens utilizam formigas artificiais com objetivo de construir as conformações para o modelo HP. Uma busca local também foi introduzida com objetivo de melhorar e manter a qualidade das soluções. 

No trabalho de Santana et al. \cite{santana2008protein} é proposto a aplicação de diferentes algoritmos de estimação de distribuição (EDA) para o PDP. Os EDAs são capazes de aprender a explorar as regularidades do espaço de busca utilizando modelos de dependência probabilísticos. Os autores compararam os resultados com as abordagens descritas anteriormente neste Capítulo e constataram que a sua abordagem conseguiu atingir os valores ótimos para várias sequências de aminoácidos.

O estudo \cite{lin2011protein} desenvolvido por Lin e Su utiliza um algoritmo genético híbrido combinando um operador de mutação baseado na otimização por exame de partículas. Os resultados apresentados por Lin et al. se mostraram superiores aos apresentados por outros estudos, da época, que utilizam algoritmos evolutivos. Este trabalho também utiliza operadores de buscas locais que serão utilizados como heurísticas de baixo nível na presente proposta. 


Custódio et al. \cite{custodio2014multiple} desenvolveram um metodologia que consistiu modificar um algoritmo genético para selecionar os operadores de cruzamento e mutação de maneira dinâmica. Além disso, utilizaram um mecanismo baseado em \textit{crowding}  para manter a diversidade durante o processo de busca. Este trabalho apresentou bons resultados em relação a outros estudos que exploram algoritmos evolutivos. Os operadores genéticos utilizados deste trabalho também serão implementados como heurísticas de baixo nível nesta proposta.

Gabriel et al. propõem um abordagem multi objetiva \cite{gabriel2012algoritmos} para resolver o PDP com modelo HP-3D. Os autores apresentam uma estratégia inovadora biobjetiva considerando duas métricas. A primeira avalia a quantidade de contatos hidrofóbicos. Já a segunda considera a distância entre os aminoácidos. Neste trabalho foi desenvolvido um AEMOs chamado AEMT \cite{gabriel2012algoritmos} algoritmo evolutivo multiobjetivo em tabelas. A formulação biobjetiva deste estudo motivou a exploração de algoritmos multiobjetivos tradicionais já aplicados com sucesso em diversos domínios de problemas.  


Misir em \cite{misir2012intelligent} apresenta o GIHH uma estratégia com objetivo de ser genérica o suficiente para ser aplicada em qualquer domínio de problema. Esta estratégia utiliza muitos mecanismos para explorar o espaço de busca. Inicialmente, uma lista Tabu é utilizada para armazenar más escolhas. O tempo de execução também é considerado afim de dar chances de execução para heurísticas que não estão sendo selecionadas com frequência. Um inteligente mecanismo de aceitação verifica se estão ocorrendo muitas iterações sem melhoria. Caso estejam a solução atual é substituída por outra solução contida em um mecanismo de memória. Este estudo obteve os melhores resultados utilizando os domínios de problemas contidos no \textit{framework} HyFlexs \cite{ochoa2012hyflex}.

Uma função escolha foi proposta em \cite{drake2012improved}. Neste estudo mecanismos de aprendizado são baseados nas melhorias obtidas pelas heurísticas de baixo nível. Um mecanismo de reforço de aprendizagem também é utilizado para atualizar os parâmetros da função dinamicamente. O critério de aceitação utilizado foi sempre aceitar toda aplicação das heurísticas de baixo nível. Este trabalho também obteve bons resultados entretanto inferiores aos resultados obtidos em \cite{misir2012intelligent}.




Lourenço et al. \cite{lourencco2012evolving} desenvolveram uma estratégia hiper-heurística utilizando evolução gramatical para geração e \textit{tuning} automático de algoritmos evolutivos. Neste trabalho uma gramática foi desenvolvida e contém os principais componentes de algoritmos evolutivos. Os resultados apresentados por Lourenço et al. provaram a habilidade da abordagem para evoluir algoritmos evolutivos. Os resultados obtidos pelos algoritmos evolutivos gerados pela evolução gramatical são competitivos com outras abordagens padrão. 


O trabalho desenvolvido por Sabar et al. \cite{sabar2015automatic} propõe uma estratégia inovadora, utilizando \text{Gene Expression Programming} (GEP), de geração de heurísticas de alto nível para um \textit {framework} hiper-heurístico aplicado a diversos problemas de \textit{benchmark} contidos no \text{framework} HyFlex \cite{ochoa2012hyflex}. Este trabalho motivou a abordagem que envolve o projeto automático de heurísticas. Os resultados apresentados se mostraram extremamente promissores. A aplicação de uma vertente de programação genética para geração de heurísticas tem uma maior capacidade de explorar espaços de busca complexos (com muitos mínimos locais) e com muitas restrições.


% A principal diferença entre as abordagns propostas  Dissertação e os outros trabalhos relacionados \cite{santana2008protein,shmygelska2002ant,shmygelska2003improved,hsu2003growth, krasnogor2002multimeme,krasnogor2002multimeme,unger1993genetic} é que este irá explorar o espaçao de busca, da mesma maneira que foi realizada no estudo \cite{sabar2015automatic}, em um nível superior: gerando heurísticas de alto nível para um \textit{framework} hiper-heurístico que será aplicado para resolver o PDP enquanto os outros trabalhos aplicam meta-heurísticas diretamente ao PDP. 




%A presente Dissertação apresenta a primeira abordagem desenvolvida para resolver PDP. A metodologia e os experimentos realizados com os AEMOs serão apresentados no prox]. Um segunda abordagem apresentada visa aplicar evolução gramatical (EG) para gerar heurísticas de alto nível para compor um \textit{framework} hiper heurístico para resolver o PDP.




\section{Considerações Finais}
\label{TrabalhosRelacionados:Conclusão}

%TODO: Melhorar essa parte
Neste Capítulo foram discutidos alguns estudos que utilizam algoritmos de busca para explorar o espaço de busca do PDP utilizando o modelo HP. Também foram discutidos trabalhos que aplicam PG como hiper-heurística de geração de heurísticas. Foram mencionadas diferentes estratégias de busca para o PDP e algumas destas estrategias servem de base para alguns componentes que esta proposta possui. Os operadores genéticos utilizados \cite{custodio2014multiple} e \cite{lin2011protein} serviram de matéria prima para as heurísticas de baixo nível desta Dissertação. O trabalho desenvolvido por \cite{sabar2015automatic} foi utilizado como base na implementação da presente Dissertação, pois obteve bons resultados dessa maneira, demonstrando a habilidade do \textit{framework} proposto generalizar bem entre diferentes domínios de problemas.

O próximo Capítulo \ref{cap:Metodologia} apresentará as duas abordagens propostas nesta Dissertação. 		% revisão bibliográfica (estado da arte)
\chapter{Metodologia}
\label{cap:Metodologia}

<<<<<<< HEAD
Neste Capítulo são apresentadas as duas estratégias propostas e desenvolvidas nesta Dissertação para o PDP simplificado. A primeira trata de uma abordagem multi-objetiva utilizando dois AEMOs. Já a segunda visa aplicação da evolução gramatical  para gerar heurísticas de alto nível para um \textit{framework} hiper-heurístico intitulada EGHyPDP.
=======
Neste Capítulo serão apresentadas as duas estratégias propostas e desenvolvidas nesta Dissertação para o PDP simplificado. A primeira trata de uma abordagem multi-objetiva utilizando dois AEMOs. Já a segunda visa aplicação da evolução gramatical  para gerar heurísticas de alto nível para um \textit{framework} hiper-heurístico intitulada EGHyPDP.
>>>>>>> 1491eafdc569b7a670d8f8b83aae192c363e6aab

Inicialmente será apresentada a representação para o problema com modelo HP-2D. Em seguida é apresentado o conjunto de heurísticas de baixo nível. Tanto a representação quanto o conjunto de heurísticas foram utilizados em ambas as abordagens propostas.

\section{Representação do PDP com modelo HP-2D}

<<<<<<< HEAD
O problema PDP simplificado foi modelado utilizando a representação relativa, descrita na subseção \ref{subsubsection:modeloHP}, afim de codificar as possíveis estruturas de proteínas em vetores de inteiros. De acordo com \cite{krasnogor1999protein} esta representação possui um maior potencial em conduzir os algoritmos a resultados melhores. Cada gene do cromossomo especifica a direção que o aminoácido atual deve ser posicionado. Cada aminoácido é posicionado na direção codificada pelo respectivo gene em relação ao aminoácido anterior. O genes podem assumir apenas 3 valores:
=======
O problema PDP simplificado foi modelado utilizando a representação relativa, descrita na subseção \ref{subsubsection:modeloHP}, afim de codificar as possíveis estruturas de proteínas em vetores de inteiros. Segundo o estudo \cite{krasnogor1999protein} esta representação possui um maior potencial em conduzir os algoritmos a resultados melhores. Cada gene do cromossomo especifica a direção que o amino ácido atual deve ser posicionado. Cada amino ácido é posicionado na direção codificada pelo respectivo gene em relação ao amino ácido anterior. O genes podem assumir apenas 3 valores:
>>>>>>> 1491eafdc569b7a670d8f8b83aae192c363e6aab

\begin{itemize}
	\item 0 indica que o próximo aminoácido deve ser posicionado à direita do aminoácido anterior
	\item 1 indica que o próximo aminoácido deve ser posicionado à frente do aminoácido anterior
	\item 2 indica que o próximo aminoácido deve ser posicionado à esquerda do aminoácido anterior.
\end{itemize}

A Figura \ref{img:cromossomo} apresenta um exemplo de um cromossomo hipotético e a conformação gerada no \textit{grid} para o modelo HP-2D.


\begin{figure}[!htb]
	\centering
	\includegraphics[scale=0.46]{Imagens/DecodedCromossome.png}
	\caption{Cromossomo decodificado que representa uma possível conformação para a cadeia HHPPHPPHHH. O primeiro e o segundo aminoácidos são fixados nas coordenadas 3,3 e 3,4 respectivamente  }
	\label{img:cromossomo}
\end{figure}


\section{Conjunto de Heurísticas de Baixo Nível}
\label{section:llhset}
Para ambas as abordagens o mesmo conjunto de heurísticas de baixo nível (operadores de cruzamento/mutação e busca locais) foi selecionado a partir dos estudos anteriores \cite{custodio2014multiple, custodio2004investigation, garza2012locality,benitez2015algoritmo}. O conjunto de heurísticas de baixo nível será descrito abaixo:

 \begin{itemize}
 	
 		\item \textit{Single Point Crossover} (1X): Esta heurística seleciona, de maneria aleatória, 1 ponto de cruzamento dividindo os indivíduos em 2 partes. Os genes entre as posições selecionadas são trocados entre os pais de modo a gerar dois novos filhos \cite{benitez2015algoritmo}.
 	
 	\item \textit{Two Points Crossover} (2X): Esta heurística seleciona, de maneria aleatória, 2 pontos de cruzamento dividindo os indivíduos em 3 partes. Os genes entre as posições selecionadas são trocados entre os pais de modo a gerar dois novos filhos \cite{benitez2015algoritmo}, conforme apresentado na Figura \ref{fig:twopointscrossover}.
 	
 	
 	\begin{figure}[!htb]
 		\centering
<<<<<<< HEAD
 		\includegraphics[scale=0.8]{Imagens/TwoPointsCrossover.png}
=======
 		\includegraphics{Imagens/TwoPointsCrossover.png}
>>>>>>> 1491eafdc569b7a670d8f8b83aae192c363e6aab
 		\caption{Exemplo de aplicação da heurística 2x. \\Fonte Autoria Própria}
 		\label{fig:twopointscrossover}
 	\end{figure}
 	
 	
 	
 	
 	\item \textit{Multi Points Crossover} (MPX): Semelhante ao 2X porém com c pontos, baseado na função $c = int(n * 0.1)$, onde $n$ é o tamanho da sequência. A heurística MPX é utilizado para promover diversidade estrutural realizando uma mescla randômica entre os pais. Embora, não tão radical quanto o \textit{Uniform  Crossover} \cite{sabar2015automatic}. Um exemplo de aplicação da heurística MPX é apresentado na imagem \ref{fig:multipointscrossover}
 	
 	
 	\begin{figure}[!htb]
 		\centering
<<<<<<< HEAD
 		\includegraphics[scale=0.8]{Imagens/MultiPointsCrossover.png}
=======
 		\includegraphics{Imagens/MultiPointsCrossover.png}
>>>>>>> 1491eafdc569b7a670d8f8b83aae192c363e6aab
 		\caption{Exemplo de aplicação da heurística MPX. \\Fonte Autoria Própria}
 		\label{fig:multipointscrossover}
 	\end{figure}
 	\item \textit{Segment Mutation} (SMUT): Altera um número aleatório (5 a 7) de genes consecutivos para direções distintas. Esta heurística introduz grandes mudanças na conformação, e tem uma grande probabilidade de criar colisões. Um mecanismo de reparação simples é aplicado no descendente gerado. A imagem \ref{fig:segmentMutation} apresenta um exemplo da aplicação do SMUT.
 	
 	\begin{figure}[!htb]
 		\centering
<<<<<<< HEAD
 		\includegraphics[scale=0.8]{Imagens/segmentMutation.png}
=======
 		\includegraphics{Imagens/segmentMutation.png}
>>>>>>> 1491eafdc569b7a670d8f8b83aae192c363e6aab
 		\caption{Exemplo de aplicação da heurística SMUT. \\Fonte Autoria Própria}
 		\label{fig:segmentMutation}
 	\end{figure}
 	
 	
 	\item \textit {Exhaustive Search Mutation} (EMUT): Esta heurística seleciona um gene aleatório e testa todas as outras direções possíveis. Manterá a alteração que conseguir aumentar a qualidade da estrutura. O \textit{tradeoff} deste operador é demandar 4 avaliações de \textit{fitness}, há mais que as demais. Esta heurística tem grande potencial de melhorar o \textit{fitness} de uma estrutura. 
 	
 	
 	\item \textit{Local Move Operator} (LM): Esta heurística troca direções entre dois genes aleatórios consecutivos. Existem algumas condições para que esta heurística possa ser executada, por exemplo, as novas direções não podem criar movimentos redundantes. A Figura \ref{fig:localMoveOperator} apresenta um exemplo da aplicação do operador LM. 
 	
 	
 	\begin{figure}[!htb]
 		\centering
<<<<<<< HEAD
 		\includegraphics[scale=0.8]{Imagens/LocalMoveOperator.png}
=======
 		\includegraphics{Imagens/LocalMoveOperator.png}
>>>>>>> 1491eafdc569b7a670d8f8b83aae192c363e6aab
 		\caption{Exemplo de aplicação da heurística LM. \\Fonte Autoria Própria}
 		\label{fig:localMoveOperator}
 	\end{figure}
 	
 	
 	\item \textit{Loop Move Operator} (LPM): Da mesma maneira que a heurística LM, esta heurística troca direções entre dois genes que estão a 5 genes de distância na sequência. A Figura  \ref{fig:loopMoveOperator} apresenta um exemplo da aplicação do operador LPM.
 	
 	
 	\begin{figure}[!htb]
 		\centering
 		\includegraphics{Imagens/LoopMoveOperator.png}
 		\caption{Exemplo de aplicação da heurística LPM. \\Fonte Autoria Própria}
 		\label{fig:loopMoveOperator}
 	\end{figure}
 	
<<<<<<< HEAD
 	\item \textit{Opposite Mutation} (OM): Esta heurística troca as direções, para direção oposta, de uma sequência de genes entre dois genes $(i,j)$ selecionados de maneira aleatória. A direção 1 ($F$) não possui oposta, portanto é mantida. Para exemplificar, suponha esta solução hipotética para uma sequência de 5 aminoácidos: $\{0,1,2,1,2\}$. Ela se tornaria $\{2,1,0,1,0\}$. A Figura \ref{fig:oppositeMutation} apresenta um exemplo da aplicação do operador OM.
=======
 	\item \textit{Opposite Mutation} (OM): Esta heurística troca as direções, para direção oposta, de uma sequência de genes entre dois genes $(i,j)$ selecionados de maneira aleatória. A direção 0 ($F$) não possui oposta, portanto é mantida. Para exemplificar, suponha esta solução hipotética para uma sequência de 5 aminoácidos: $\{0,1,2,1,2\}$. Ela se tornaria $\{0,2,1,2,1\}$. A Figura \ref{fig:oppositeMutation} apresenta um exemplo da aplicação do operador OM.
>>>>>>> 1491eafdc569b7a670d8f8b83aae192c363e6aab
 	
 	
 	\begin{figure}[!htb]
 		\centering
<<<<<<< HEAD
 		\includegraphics[scale=0.8]{Imagens/OppositeMutation.png}
=======
 		\includegraphics{Imagens/OppositeMutation.png}
>>>>>>> 1491eafdc569b7a670d8f8b83aae192c363e6aab
 		\caption{Exemplo de aplicação da heurística OM. \\Fonte Autoria Própria}
 		\label{fig:oppositeMutation}
 	\end{figure}
 	
 	
 	
 \end{itemize} 



<<<<<<< HEAD
 Este Capítulo está divido em duas seções para melhor apresentar ambas as estratégias propostas nesta Dissertação. Na Seção \ref{sec:aemos} foi apresentado a estratégia multi objetiva, a qual utiliza dois algoritmos evolucionários, descritos na literatura para otimização multi objetiva. Já a Seção  \ref{sec:eghypdp} irá apresentar o design automático de heurísticas de alto nível para um \textit{framework} hiper heurístico para resolver o PDP.
=======
 Este Capítulo está divido em duas seções para melhor apresentar ambas as estratégias propostas nesta Dissertação. A Seção \ref{sec:aemos} apresenta a estratégia multi objetiva onde foi utilizada dois algoritmos evolucionários, do estado da arte de otimização multi objetiva. Já a Seção  \ref{sec:eghypdp} irá apresentar o design automático de heurísticas de alto nível para um \textit{framework} hiper heurístico para resolver o PDP.
>>>>>>> 1491eafdc569b7a670d8f8b83aae192c363e6aab

	



\section{AEMOs aplicados ao PDP}
\label{sec:aeoms}

Esta abordagem utiliza uma modelagem multi-objetiva para PDP baseado no estudo desenvolvido em \cite{gabriel2012algoritmos}. O primeiro objetivo consiste em maximizar a quantidade de contatos topológicos das estruturas de proteínas. Já o segundo  trata de minimizar a máxima distância euclidiana entre os aminoácidos. 

<<<<<<< HEAD
Duas abordagens multi-objetivas foram desenvolvidas neste Capítulo, utilizando os AEMOs (NSGAII e IBEA) descritos no Capítulo \ref{cap:Referencial Teórico}. A primeira abordagem consistiu em aplicar os algoritmo IBEA and NSGAII utilizando suas versões padrão. Nesta abordagem os operadores (heurísticas de baixo nível) genéticos (cruzamento e mutação) são fixos com: \textit{Single Point Crossover (1x)} and \textit{Bit Flip Mutation (BM)}. Esta foi a combinação que obteve os melhores resultados em experimentos preliminares. No caso da segunda abordagem, o IBEA e NSGAII foram modificados com objetivo de aprimorar os resultados em relação às versões padrões. Duas modificações foram incluídas e serão descritas abaixo:
=======
Duas abordagens multi-objetivas foram desenvolvidas neste Capítulo, utilizando os AEMOs (NSGAII e IBEA) descritos no Capítulo \ref{cap:Referencial Teórico}. A primeira abordagem consistiu em aplicar os algoritmo IBEA and NSGAII utilizando suas versões padrão. Nesta abordagem os operadores (heurísticas de baixo nível) genéticos (cruzamento e mutação) são fixos com: \textit{Single Point Crossover (1x)} and \textit{Bit Flip Mutation (BM)}. Esta foi a combinação que obteve os melhores resultados em experimentos preliminares. No caso da segunda abordagem, o IBEA e NSGAII foram modificados com objetivo de aprimorar os resultados em relação às versões padrões. Duas modificações foram propostas e serão descritas abaixo:
>>>>>>> 1491eafdc569b7a670d8f8b83aae192c363e6aab

 
 \begin{itemize}
 		
		\item \textit Conjunto de heurísticas de baixo nivel: O uso de operadores fixos geralmente não conseguem guiar a busca para regiões promissoras. 
		Com objetivo de aprimorar os AEMOs, um conjunto de heurísticas foi proposto. As heurísticas que compõem conjunto foram selecionados de estudos anteriores e foram apresentados no início deste Capítulo. A cada operação de cruzamento e mutação as heurísticas são selecionados de maneira aleatória a partir do conjunto. As heurísticas são sempre executadas independente de probabilidades conforme a versão padrão dos algoritmos. 
	
		
		\item Inicialização via \textit{backtracking}: Tradicionalmente, a população inicial é gerada de maneira aleatória no caso dos algoritmos NSGAII e IBEA. Este tipo de inicialização tem grande potencial de gerar muitas soluções inválidas ao  modelo HP-2D. Soluções que não sejam \textit{self-avoiding walk} (SAW) são consideradas inválidas pois dois ou mais aminoácidos estariam ocupando a mesma posição no espaço. Se a população for integralmente gerada de maneira aleatória os algoritmos de otimização perdem um tempo considerável avaliando soluções inválidas. Para evitar este problema uma estratégia de \textit{backtracking} pode ser utilizada. A estratégia de inicialização com \textit{backtracking} irá começar posicionando o primeiro aminoácido na posição 0,0. Para posicionar o próximo aminoácido, um movimento é selecionado de maneira aleatória. Caso o movimento cause uma colisão, este movimento será marcado como uma má escolha e um novo movimento é selecionado aleatoriamente (do conjunto que restou sem os movimentos marcados como más escolhas). Caso todos os movimentos estejam marcados como má escolha, a estratégia de \textit{backtracking} irá retornar de maneira recursiva para o aminoácido anterior e marcar a escolha em questão como uma má escolha. A estratégia de \textit{backtracking} termina quando gerar uma conformação que não possua colisões. Entretanto, a inicialização via \textit{backtracking} é computacionalmente custosa. Dessa maneira, baseado no trabalho de \cite{benitez2015algoritmo}, apenas 20\% da população inicial foi inicializada utilizando esta estratégia.

\end{itemize}

Portanto 4 algoritmos foram implementados IBEA, NSGAII, M\_IBEA e M\_NSGAII foram propostos para avaliar a abordagem multi objetiva para o PDP simplificado.


\subsection{Funções Objetivo}


\begin{itemize}
	\item \textbf{Valor de Energia}: Este é o objetivo principal e sua responsabilidade é avaliar o valor de energia associado com as possíveis conformações codificadas pelos cromossomos. O objetivo é minimizar o valor de energia, o qual, é calculado conforme descrito no Capítulo \ref{cap:pdp}. Este objetivo guia a busca na direção onde os valores energia associados com as estruturas de proteínas sejam mínimos. Dessa maneira, obtendo conformações mais próximas ao estado nativo das estruturas de proteínas.

    \item \textbf{Distância euclideana entre os resíduos mais distantes}: Este é um objetivo secundário inspirado pelo estudo desenvolvido por \cite{gabriel2012algoritmos}. A motivação por de trás deste objetivo é que estruturas mais compactas tendem a possuir mais contatos hidrofóbicos, oque resultaria em um valor menor de energia. A distância entre os resíduos é calculada utilizando a distância euclidiana.
   
\end{itemize}

Geralmente para avaliar e comparar a performance dos AEMOs, indicadores de qualidade são utilizados. Neste estudo o indicador \textit{hypervolume} normalizado foi utilizado. Este indicador considera o volume do espaço de busca dominado pela fronteira conhecida de Pareto obtida por um algoritmo \cite{zitzler2003performance}. Um maior valor de \textit{hypervolume} significa maior qualidade na cobertura do que um algoritmo com valor inferior.

Os 4 algoritmos foram implementados utilizando a arquitetura \textit{open source} disponível no \textit{framework} jMetal. A arquitetura do jMetal é de fácil extensão e possui uma comunidade ativa	.


\section{EGHyPDP}
\label{sec:eghypdp}

Esta abordagem é baseada no trabalho desenvolvido por \cite{sabar2015automatic}, o qual  utilizou GEP (\textit{gene expression programming}) com objetivo de gerar, de maneira \textit{online}, os componentes de um \textit{framework} hiper-heurístico para diversos domínios de problemas. Os testes de generalidade realizados, utilizando os 6 domínios providos pelo \textit{framework} hiper-heurístico HyFlex, apresentaram bons resultados em relação às outras estratégias hiper-heurísticas do estado da arte. Nesta proposta pretende-se utilizar EG ao invés de GEP e aplicar ao PDP simplificado utilizando o modelo HP-2D. Da mesma maneira quando utilizando  AEMOs a representação de coordenadas relativas descrita na subseção \ref{subsubsection:modeloHP}, será utilizada. Como mencionado anteriormente, um \textit{framework} hiper-heurístico possui dois níveis: alto (\textit{high-level heuristics}) e baixo (\textit{low-level heuristics}). Nesta proposta as heurísticas de alto nível são compostas por: um mecanismo de seleção e um critério de aceitação. Já as heurísticas de baixo nível consistem em um conjunto de heurísticas, selecionadas de estudos anteriores, um mecanismo de memória e uma função de \textit{fitness}. 

\section{Heurísticas de alto nível}
\label{sec:highlevelheuristics}
Esta abordagem  foi desenvolvida para construir de maneria \textit{offline} os componentes de uma heurística de alto nível (mecanismo de seleção e critério de aceitação) para compor um \textit{framework} hiper-heurístico. A Figura \ref{fig:proposedFramework} apresenta a estrutura geral do EGHyPDP. 

\begin{figure}[!htb]
	\centering
	\includegraphics[scale=.98]{Imagens/proposedFramework.png}
	\caption{ \textit{Estrutura geral do EGHyPDP.} \\ Fonte: Adaptado de \cite{sabar2015automatic}}
	\label{fig:proposedFramework}
\end{figure}


<<<<<<< HEAD
Heurísticas de alto nível geralmente levam em consideração uma ou mais informações referentes ao histórico das aplicações das heurísticas de baixo nível para tomar suas decisões. Tradicionalmente, informações tais como desempenho (capacidade de melhorar soluções), tempo (desde a última aplicação de uma dada heurística) e intervalo de confiança) são utilizadas como base de conhecimento.  Sabar et al. \cite{sabar2015automatic} propõem a utilização de vários critérios para avaliar as heurísticas de baixo nível. Cada critério irá favorecer a seleção de uma heurística de baixo nível a partir de um aspecto diferente. Por exemplo, algumas heurísticas de baixo nível podem ter bom desempenho apenas no início da busca, enquanto outras podem obter melhores resultados apenas ao final. Estes critérios propostos por Sabar et al. contém estatísticas referente à aplicações das heurísticas de baixo nível e são genéricos o suficiente para serem aplicados ao PDP. Os critérios propostos em \cite{sabar2015automatic} são detalhados em seguida:
=======
Heurísticas de alto nível geralmente levam em consideração uma ou mais informações referentes ao histórico das aplicações das heurísticas de baixo nível para tomar suas decisões. Tradicionalmente, informações tais como desempenho (capacidade de melhorar soluções), tempo (desde a última aplicação de uma dada heurística) e intervalo de confiança (no caso de estratégias que utilizam MAB) são utilizadas como base de conhecimento.  Sabar et al. \cite{sabar2015automatic} propõem a utilização de vários critérios para avaliar as heurísticas de baixo nível. Cada critério irá favorecer a seleção de uma heurística de baixo nível a partir de um aspecto diferente. Por exemplo, algumas heurísticas de baixo nível podem ter bom desempenho apenas no início da busca, enquanto outras podem obter melhores resultados apenas ao final. Estes critérios propostos por Sabar et al. contém estatísticas referente à aplicações das heurísticas de baixo nível e são genéricos o suficiente para serem aplicados ao PDP. Os critérios propostos em \cite{sabar2015automatic} são detalhados em seguida:
>>>>>>> 1491eafdc569b7a670d8f8b83aae192c363e6aab


\begin{itemize}
	\item RC (\textit{Reward Credit}): Representa a recompensa que uma determinada heurística de baixo nível deve receber baseado no seu desempenho durante o processo de busca. Quando a i-ésima heurística é aplicada, a melhoria para a solução é computada. O cálculo da melhoria é dado por: $M(i) = (|f1 -f2|/f1) *100$ se $f2$< $f1$, onde $f1$ é a qualidade da solução corrente e $f2$ é a qualidade da solução resultante após a aplicação da i-ésima heurística. 
	A melhoria obtida é salva em uma janela deslizante (FIFO) de tamanho W. O crédito de qualquer heurística de baixo nível é então atribuído como o máximo valor na janela deslizante correspondente. A ideia por trás deste critério é: heurísticas de baixo nível que não são usadas com frequência mas que alteram a solução com grandes melhorias tendem a ter mais preferência do que aquelas que geram pequenas melhorias. Portanto as heurísticas que trazem frequentes, mas pequenas melhorias irão ter menos probabilidade de serem selecionadas.
	\item $C_{best}$: Número de vezes que a i-ésima heurística de baixo nível atualizou a melhor solução conhecida. Este critério favorece as heurísticas de baixo nível que obtiveram êxito em melhorar a melhor solução conhecida até o momento. Este critério é útil para sistematicamente melhorar o atual mínimo local.
	\item $C_{current}$: Número de vezes que a i-ésima heurística de baixo nível atualizou a solução atual. Este critério favorece as heurísticas de baixo nível que obtém êxito em atualizar a solução corrente. Este critério serve para deixar a busca concentrada próxima à solução corrente.
	\item $C_{accept}$: Número de vezes que a solução gerada pela i-ésima heurística de baixo nível foi aceita pelo critério de aceitação. Irá favorecer heurísticas de baixo nível que podem ajudar a escapar de um mínimo local.
	\item $C_{ava}$: A média de melhorias anteriores da i-ésima heurística de baixo nível durante o progresso da busca. Este critério favorece heurísticas de baixo nível que realizaram grandes melhorias em média.
	\item $C_r$: O número de vezes que a i-ésima heurística de baixo nível foi classificada como primeira.  
\end{itemize} 

Da mesma maneira Sabar et al. propõem o uso de dados referentes ao histórico de aplicações das heurísticas de baixo nível para compor critérios de aceitação que irão definir limites para aceitar soluções com qualidade inferior. Dessa forma, um conjunto de fatores também foi proposto e será detalhado em seguida:


\begin{itemize}
	\item Delta: A diferença da qualidade entre a solução corrente e a solução descendente.
	\item PF: A qualidade da solução anterior.
	\item CF: A qualidade da solução atual.
	\item CI: Iteração corrente.
	\item TI: Número de iterações.
\end{itemize}


Utilizando estes dados estatísticos e um conjunto de funções matemáticas simples, tais como soma, subtração, multiplicação e divisão, uma gramática foi desenvolvida, durante este estudo, para suportar a geração das heurísticas de alto nível. A gramática desenvolvida para gerar mecanismos de seleção e critérios de aceitação é apresentada na Gramática \ref{grammar:proposedGrammar}. 

Para inicializar os dados dos terminais: todas as heurísicas foram executadas uma vez e os dados para cada terminal foi calculado. Toda iteração seguinte irá atualizar os dados dos terminais e essas informações são utilizadas durante a busca.

 \begin{Grammar}
 	\begin{grammar}
 		<hh-selection> ::= <selection-mechanism> <acceptance-criterion> 
 		
 		<selection-mechanism> :==  <selection-terminal>   
 		\alt <selection-mechanism> <math-function> <selection-mechanism> 
 		\alt (<selection-mechanism> <math-function> <selection-mechanism>) 
 		
 		<selection-terminal> :== 
 		RC 
 		| Cbest 
 		| Ccurrent 
 		| Caccept 
 		| Cava 
 		| Cr
 		
 		<math-function> :== + 
 		| - 
 		| * 
 		| \%
 		
 		<acceptance-criterion> ::== <acceptance-terminal> 
 		\alt <acceptance-criterion> <math-function>
 		<acceptance-criterion>
 		\alt (<acceptance-criterion>  <math-function> <acceptance-criterion>) 
 		
 		<acceptance-terminal> :== PF | CF | CI | TI
 		
 		%	<acceptance-function> :== + | - | * | \% | $e^x$
 		
 		
 	\end{grammar}
 	\caption{Gramática definida para gerar  heurísticas de alto nível}
 	\label{grammar:proposedGrammar}
 \end{Grammar}
 
 
  
  O conjunto de funções matemáticas para combinar de diferentes maneiras os dados históricos das aplicações das heurísticas de baixo nível é apresentado abaixo:
  
  \begin{itemize}
  	\item +: Adiciona as duas entradas.
  	\item -: Subtrai a segunda entrada da primeira.
  	\item *: Multiplica as duas entradas.
  	\item \%: Divisão protegida, isto é, se o denominador for 0, o altera para 0,001.
  \end{itemize}
  
  
  Utilizando a Gramática \ref{grammar:proposedGrammar} e vetores de inteiros é possível gerar heurísticas de alto nível. Os conjuntos terminais da gramática apresentam estatísticas sobre as heurísticas de baixo nível e estas são a matéria-prima para a construção dos componentes das heurísticas de alto nível de um \textit{framework} hiper-heurístico. 
  
  

  
  O próximo passo consiste em evoluir uma população de vetores de inteiro utilizando o processo evolutivo descrito na subseção 
<<<<<<< HEAD
	  \ref{subsubsection:EvolucaoGramatical}. %A Figura BLAH apresenta o processo geral da evolução gramatical proposta.
=======
  \ref{subsubsection:EvolucaoGramatical}. %A Figura BLAH apresenta o processo geral da evolução gramatical proposta.
>>>>>>> 1491eafdc569b7a670d8f8b83aae192c363e6aab
  
  
  \subsection{Função de \textit{Fitness}}
  \label{sub:funcfitness}
  
  
  	%TODO: escrever na metodologia sobre a funao de fitness escrever uma funcao bunitinha 
  	
 Com objetivo de avaliar os indivíduos gerados durante a busca, uma função de \textit{fitness} foi desenvolvida. A função executa a heurística de alto nível, representada por um dado indivíduo, 
 em 1/4 do total instâncias utilizadas como \textit{benchmark}. Conforme será descrito no Capítulo \ref{cap:experimentos} foram   foram selecionadas 11 instâncias de diversos estudos que exploram o PDP simplificado com modelo HP-2D. Portanto 1/4 de 11 resulta em 3 instâncias que serão selecionadas de maneira aleatória. Para cada instância a heuristíca de alto nível foi executada com um tempo máximo de 30 minutos e o retorno é a melhor solução para o modelo HP-2D. O valor de \textit{fitness} associado com a solução retornada é então normalizado entre 0 e 1. O \textit{fitness} de um indivíduo, da EG, consiste na soma das saídas das execuções com cada uma das 3 instâncias. Dessa maneira, o melhor valor possível é 3 e o pior é 0. A razão de executar a heurística de alto nível (indivíduo) com 1/4 do total de instâncias é tornar as heurísticas geradas mais genéricas ao invés de especializada em apenas uma instância. 
  
 
% 
%\begin{equation}
% 	\sum_{i = 0}^{n}E(c_i)
% \end{equation}
%
%\noindent onde $ c $ é o conjunto de instâncias selecionadas de maneira aleatória no inicio do processo da EGHyPDP, sendo que o tamanho máximo, $ n $ foi definido como três.
 
 
 
   


  
  \subsection{Critério de Parada}
  \label{sub:criterioParada}
  
  Para terminar o processo da EG um número máximo de iterações que não obtêm melhora será utilizado como condição de parada. Note que este critério de parada é referente à parada do processo da EG e não das execuções dos indivíduos dentro do \textit{framework} hiper-heurístico, que ocorrem durante o progresso da EG. 
  
  
  \section{Heurísticas de baixo nível}
  
  Nas heurísticas de baixo nível o EGHyPDP possui 2 componentes principais: o conjunto de heurísticas de baixo nível utilizado o qual foi apresentado na seção \ref{section:llhset} e um mecanismo de memória o qual será descrito na subseção \ref{sub:MecanismoDeMemoria}.
  



\section{Processo geral do EGHyPDP} 

As principais etapas da EGHyPDP proposta serão apresentadas nesta Seção.
Inicialmente uma população de indivíduos (heurísticas de alto nível: mecanismos de seleção e critérios de aceitação) é gerada conforme o procedimento que será descrito posteriormente nesta Seção. O \textit{fitness} da população é calculado inserindo os indivíduos em um \textit{framework} hiper-heurístico e o executando com 3 instâncias por 30 minutos. E de maneira iterativa selecionar indivíduos pais e aplicar os operadores de cruzamento, \textit{prune}, mutação, e \textit{duplicate} para gerar descendentes. Posteriormente, estes indivíduos são submetidos ao processo de avaliação descrito na subseção \ref{sub:funcfitness}.

%Para avaliar os indivíduos gerados, os seguintes passos são executados:


O processo da EGHyPDP irá parar apenas quando o critério de parada discutido na subseção \ref{sub:criterioParada} for atingido e será retornado o indivíduo (heurística de alto nível) que possuir o maior valor de \textit{fitness}. Também será retornada a solução ao PDP que tiver maior qualidade no mecanismo de memória.



\subsection{Mecanismo de Memória}
\label{sub:MecanismoDeMemoria}

A maioria dos \textit{frameworks} hiper-heurísticos propostos na literatura operam sobre uma única solução \cite{chakhlevitch2008hyperheuristics, burke2013hyper}. Blum et al. \cite{blum2011hybrid} menciona que utilizar uma única solução pode restringir a capacidade de explorar complexos espaços de busca e com alta variância de características. Dessa maneira,  \cite{sabar2015automatic} propôs uma abordagem que utiliza um mecanismo de memória, assim como  \cite{talbi2006cosearch}, o qual contém um conjunto de soluções com alta qualidade e diversificadas, atualizado durante o progresso da busca. Nesta proposta o mecanismo de memória tem a responsabilidade de armazenar soluções para o problema PDP utilizando a representação de coordenadas relativas para o modelo HP-2D. 

\subsubsection{Inicialização do Mecanismo de Memória}

<<<<<<< HEAD
Tradicionalmente algoritmos evolutivos inicializam suas populações de maneira aleatória, por conta disto  muitas soluções inválidas ao modelo HP são geradas na inicialização. Isto geralmente  ocasiona perda de tempo de processamento, por conta da grande quantidade de conformações inválidas antes que bons resultados sejam obtidos. Diante disto, \cite{benitez2015algoritmo} propôs uma estratégia especializada de inicialização. A população de seu algoritmo genético, é dividia em duas partes. Uma gerada aleatoriamente, com indivíduos que potencialmente possuem colisões. E uma segunda parte onde todos os indivíduos são livres de colisões. Uma configuração é utilizada para definir a proporção entre as duas partes da população inicial. Para garantir que os indivíduos não possuam colisões, uma estratégia de \textit{backtracking} deve ser utilizada. Nesta abordagem, a mesma estratégia de inicialização via \textit{backtracking }, descrita  na Seção \ref{sec:aeoms} foi implementada.
=======
Tradicionalmente algoritmos evolutivos inicializam suas populações iniciais de maneira aleatória, por conta disto  muitas soluções inválidas ao modelo HP, são geradas na inicialização. Isto geralmente  ocasiona perda de tempo de processamento, por conta da grande quantidade de conformações inválidas antes que bons resultados sejam obtidos. Diante disto, \cite{benitez2015algoritmo} propôs uma estratégia especializada de inicialização. A população de seu algoritmo genético, é dividia em duas partes. Uma gerada aleatoriamente, com indivíduos que potencialmente possuem colisões. E uma segunda parte onde todos os indivíduos são livres de colisões. Uma configuração é utilizada para definir a proporção entre as duas partes da população inicial. Para garantir que os indivíduos não possuam colisões, uma estratégia de \textit{backtracking} deve ser utilizada. Nesta abordagem, a mesma estratégia de inicialização via \textit{backtracking }, descrita  na Seção \ref{sec:aeoms} foi implementada.
>>>>>>> 1491eafdc569b7a670d8f8b83aae192c363e6aab





%As possíveis conformações podem ser representadas por um caminho em um grafo orientado estruturado como uma árvore. Consequentemente, cada nó da árvore representa uma solução candidata parcial $c$, desde o primeiro aminoácido até o último sendo considerado. Portanto, um caminho até um nó folha representa uma conformação completa. As arestas do grafo representam o movimento de cada aminoácido relativo a seu predecessor.



\subsubsection{Atualização do Mecanismo de Memória}
Para cada indivíduo (heurística de alto nível) será selecionada de maneira aleatória uma solução do mecanismo de memória e a busca irá iniciar em torno desta solução, quando o \text{framework} hiper-heurístico atingir o seu número máximo de iterações a solução final tem que ser avaliada para verificar sua qualidade e diversidade. A qualidade de uma solução para o PDP utilizando o modelo HP é inversamente proporcional à quantidade de interações entre aminoácidos hidrofóbicos. Portanto a qualidade de uma solução é dada pela quantidade de iterações H-H multiplicada por -1, conforme descrito na subseção \ref{subsubsection:modeloHP}.  As soluções geradas que tiverem a qualidade maior que todas as soluções contidas no mecanismo de memória substituirão a solução que tiver menor similaridade segundo a distância de  \cite{hamming1950error}. Se a qualidade de uma solução gerada não for maior que todas as soluções, mas melhor em relação a um sub-conjunto do mecanismo de memória, esta substituirá a solução que tiver menor qualidade e menor similaridade do sub-conjunto. E por fim se a qualidade da solução gerada for pior que todas contidas no mecanismo de memória, esta é descartada. A similaridade é considerada a fim de manter a diversidade entre as soluções.  



\section{Considerações Finais}
\label{Metodologia:ConsideracoesFinais}

<<<<<<< HEAD
Neste Capítulo foram discutidos os principais aspectos relativos à duas estratégias propostas nesta Dissertação. Inicialmente, foi discutido sobre a aplicação de AEMOs e em seguida o EGHyPDP foi apresentado. A representação relativa foi utilizada para codificar as soluções para ambas as estratégias. O conjunto de heurísticas de baixo nível, utilizado em ambas abordagens, também foi apresentado. No caso dos AEMOs 4 versões de algoritmos foram apresentadas. As funções objetivo também foram discutidas. Duas adaptacões foram propostas que consistiram em adicionar um conjunto heurísticas de baixo nível para tornar os AEMOs mais adpatativos e a inicialização via \textit{backtracking}. Para aplicar o EGHyPDP com objetivo de gerar heurísticas de alto nível de um \text{framework} hiper-heurístico para o PDP simplicado, uma  gramática foi desenvolvida. Também foi apresentado um conjunto de terminais referente ao histórico das aplicações das heurísticas de baixo nível. Operações aritméticas compõem a gramática e combinadas com os dados estatísticos é possível criar diferentes heurísticas de alto nível. Posteriormente, a função de \textit{fitness} avalia estas heurística executando o \textit{framework} composto por tais três vezes em três instâncias distintas selecionadas de maneira aleatória.
=======
Neste Capítulo foram discutidos os principais aspectos relativos à duas estratégias propostas nesta Dissertação. Inicialmente, foi discutido sobre a aplicação de AEMOs e em seguida o EGHyPDP foi apresentado. A representação relativa foi utilizada para codificar as soluções para ambas as estratégias. O conjunto de heurísticas de baixo nível, utilizado em ambas abordagens, também foi apresentado. No caso dos AEMOs 4 versões de algoritmos foram apresentadas. As funções objetivas também foram discutidas. Duas adaptacões foram propostas que consistiram em adicionar um conjunto heurísticas de baixo nível para tornar os AEMOs mais adpatativos e a inicialização via \textit{backtracking}. Para aplicar o EGHyPDP com objetivo de gerar heurísticas de alto nível de um \text{framework} hiper-heurístico para o PDP simplicado, uma  gramática foi desenvolvida. Também foi apresentado um conjunto de terminais referente ao histórico das aplicações das heurísticas de baixo nível. Funções matemáticas simples compõem a gramática e combinadas com os dados estatísticos diferentes heurísticas de alto nível foram geradas. Posteriormente, a função de \textit{fitness} avalia estas heurística executando o \textit{framework} composto por tais trés vezes em três instâncias distintas selecionadas de maneira aleatória.
>>>>>>> 1491eafdc569b7a670d8f8b83aae192c363e6aab
 
 O próximo Capítulo apresenta os experimentos realizados afim de avaliar o desempenho das abordagens aqui propostas.



		% proposta
\chapter{Experimentos}

Neste capítulo serão apresentados os experimentos para validar a duas estratégias apresentadas no Capítulo \ref{cap:Metodologia}. 


Este capítulo esta divido em duas seções para melhor apresentar a duas estratégias propostas. A primeira seção descreverá os experimentos realizados para avaliar a aplicação de AEMOs para o PDP utilizando o modelo HP-2D. A segunda seção apresenta os experimentos realizados para avaliar a habilidade do EGHyPDP para gerar heurísticas de alto nível para um \textit{framework} hiper heurístico, utilizando evolução gramatical. 

\section{Resultados dos AEMOs aplicados ao PDP}

Nesta seção serão apresentados o cojunto de experimentos realizados utilizando a abordagem multi-objetiva com os algoritmos NSGAII e IBEA. Inicialmente, serão apresentadas as configurações utilizadas nos algoritmos. Em seguida, uma comparação entre os resultados de cada algoritmo é apresentada. Finalmente, uma comparação com os resultados dos trabalhos relacionados que tratam o problema de maneira mono-objetiva é apresentada.




A configuração utilizada nos AEMOs foi definida baseada no tamanho das sequência. Para sequências menores o tamanho da população e a numero máximo de avaliações são menores. Já para as sequências maiores os valores utilizados são superiores. A Tabela \ref{tab:popConfiguration} apresenta para cada sequência: o tamanho da sequência, o tamanho da população e número máximo de avaliações.TODO: ver como colocar essa aprte, tem que explicar as abordagens na metodologia In the case of the first approach, the probability of crossover/mutation occurrence was fixed, for all sequences, to 0.9 and 0.01 respectively. The second approach does not use a probability since the operators are always applied to generate new individuals. 
No caso do algoritmo IBEA a população auxiliar foi mantida em 200 para todas as sequências. Para cada sequência os algoritmos foram executados 30 vezes de maneira independente.


\begin{table}[!htb]
	\centering
	\caption{Tamanho da população, número máximo de avaliações para cada sequência}
	\label{tab:popConfiguration}
	\begin{tabular}{ccccl}
		\hline
		Sequência & Tamanho & \begin{tabular}[c]{@{}c@{}}Tamanho  \\ 	População\end{tabular} & \begin{tabular}[c]{@{}c@{}}N Max Avaliações \\ \end{tabular} & \multicolumn{1}{c}{} \\ \hline
		sq1      & 20   & 100                                                        & 25000                                                      &                      \\ \hline
		sq2      & 24   & 100                                                        & 25000                                                      &                      \\ \hline
		sq3      & 25   & 500                                                        & 250000                                                     &                      \\ \hline
		sq4      & 36   & 500                                                        & 250000                                                     &                      \\ \hline
		sq5      & 48   & 1000                                                       & 2500000                                                    &                      \\ \hline
		sq6      & 50   & 1000                                                       & 2500000                                                    &                      \\ \hline
		sq7      & 60   & 2500                                                       & 2500000                                                    &                      \\ \hline
		sq8      & 64   & 2500                                                       & 2500000                                                    &                      \\ \hline
	\end{tabular}
\end{table}


Conforme mencionado no capítulo \ref{cap:Metodologia} o indicador \textit{hypervolume} foi utilizado para comparar o desempenho dos AEMOs. Para cada algoritmo a média e o desvio padrão do \textit{hypervolume}  referente a 30 execuções é apresentada na Tabela \ref{tab:hypervolumeResults}. Os maiores valores estão em negrito e indicam uma maior aproximação da fronteira de Pareto TODO: rerencia pareto.

Observando a Tabela \ref{tab:hypervolumeResults} é possível notar que, exceto para $sq1$, para todas outras sequências o algoritmo M\_IBEA (versão modificada \textit{backtrack} e um \textit{pool} de operadores) obteve a maior média de \textit{hypervolume}. No caso da $sq1$, o IBEA sem modificações obteve a média mais alta. Também vale mencionar, que quando comparando apenas o NSGAII e o M\_NSGAII, a versão modificada obteve as melhores médias. De maneira geral, os AEMOs com \textit{backtrack} e o \textit{pool} de operadores (M\_IBEA and M\_M_NSGAII) apresentaram melhoria considerável em relação aos AEMOs tradicionais. Na Tabela \ref{tab:hypervolumeResults}, as células do M\_IBEA que estão marcados de cinza apresentaram diferença estatística de acordo com o teste de  Kruskal-Wallis \cite{mckight2010kruskal} entre M\_IBEA e todos os outros algoritmos (NSGAII, M\_NSGAII e IBEA).


\begin{table}[]
	\centering
	\caption{Resultado de média/desvio padrão dos AEMOs}
	\label{tab:hypervolumeResults}
	\begin{tabular}{|c|c|c|c|c|}
		\hline
		\multirow{2}{*}{Sequência} & \multicolumn{4}{c|}{\begin{tabular}[c]{@{}c@{}}\textit{Hypervolume} Média\\ (Desvio padrão)\end{tabular}} \\ \cline{2-5} 
		& NSGAII            & M\_NSGAII         & IBEA                     & M\_IBEA                 \\ \hline
		\multirow{2}{*}{sq1}      & 0.742827          & 0.720864          & \textbf{0.789712}        & 0.786571       \\
		& (0.106315)        & (0.131351)        & (0.067660)               & (0.099424)              \\ \hline
		\multirow{2}{*}{sq2}      & 0.680572          & 0.712275          & 0.719960                 & \textbf{0.737086}       \\
		& (0.083445)        & (0.137226)        & (0.080727)               & (0.095299)              \\ \hline
		\multirow{2}{*}{sq3}      & 0.671171          & 0.709898          & 0.716438                 & \textbf{0.738017}       \\
		& (0.129417)        & (0.124201)        & (0.148112)               & (0.155638)              \\ \hline
		\multirow{2}{*}{sq4}      & 0.702280          & 0.740153          & 0.751755                 & \textbf{0.785728}       \\
		& (0689832)         & (0.075271)        & (0.092427)               & (0.055607)              \\ \hline
		\multirow{2}{*}{sq5}      & 0.707654          & 0.758128          & 0.733464                 & \cellcolor[HTML]{C0C0C0}\textbf{0.807637}       \\
		& (0.082611)        & (0.062315)        & (0.128757)               & \cellcolor[HTML]{C0C0C0}(0.039620)              \\ \hline
		\multirow{2}{*}{sq6}      & 0.667771          & 0.774017          & 0.728699                 & \cellcolor[HTML]{C0C0C0}\textbf{0.821177}       \\
		& (0.132218)        & (0.063231)        & (0.080679)               & \cellcolor[HTML]{C0C0C0}(0.048124)              \\ \hline
		\multirow{2}{*}{sq7}      & 0.784483          & 0.792843          & 0.801778                 & \textbf{0.810351}       \\
		& (0.063257)        & (0.033062)        & (0.067111)               & (0.054576)              \\ \hline
		\multirow{2}{*}{sq8}      & 0.677464          & 0.705798          & 0.7450656                & \cellcolor[HTML]{C0C0C0}\textbf{0.811439}       \\
		& (0.041287)        & (0.053048)        & (0.036454)               & \cellcolor[HTML]{C0C0C0}(0.050087)              \\ \hline
	\end{tabular}
\end{table}


\subsection{Comparação com outras abordagens mono-objetivas}
Esta subseção apresenta a comparação dos resultados obtidos pelas melhores variantes de cada AEMO com outras abordagens mono-objetivas propostas em trabalhos anteriores. Os trabalhos comparados são: GA \cite{unger1993genetic}, MMA \cite{krasnogor2002multimeme}, ACO \cite{shmygelska2002ant},  NewACO \cite{ shmygelska2003improved} e PERM \cite{hsu2003growth}. Todos os trabalhos mencionados tratam-se de abordagens mono-objetivas, considerando apenas o valor de energia referente as estruturas de proteínas. A Tabela \ref{tab:comparison} apresenta os melhores resultados, em termos da energia, pelas versões modificadas dos AEMOs, assim como os melhores resultados obtidos pelos estudos anteriores.




\begin{table}[htb]
	\centering
	\caption{Comparação dos melhores AEMOs com o estudos anteriores do PDP}
	\label{tab:comparison}
	\begin{tabular}{ccccccccc}
		\hline
		inst                    & M\_IBEA      & M\_NSGAII    & \begin{tabular}[c]{@{}c@{}}EDA \\ 
		
		\end{tabular}           & 
		
		\begin{tabular}[c]{@{}c@{}}GA \\ 
		
		\end{tabular}   
		
		& 	\begin{tabular}[c]{@{}c@{}} MMA \\ 
			  
		\end{tabular}        
		&
		\begin{tabular}[c]{@{}c@{}}  ACO \\ 
			
		\end{tabular} 
		& 
		\begin{tabular}[c]{@{}c@{}}  NewACO\\ 
			
		\end{tabular}
		& 
		\begin{tabular}[c]{@{}c@{}}  PERM\\ 
			 
		\end{tabular}
		\\ \hline
		sq1                     & \textbf{-9}  & \textbf{-9}  & \textbf{-9}  & \textbf{-9}  & \textbf{-9}  & \textbf{-9}  & \textbf{-9}  & \textbf{-9}  \\ \hline
		sq2                     & \textbf{-9}  & \textbf{-9}  & \textbf{-9}  & \textbf{-9}  & \textbf{-9}  & \textbf{-9}  & \textbf{-9}  & \textbf{-9}  \\ \hline
		sq3                     & \textbf{-8}  & \textbf{-8}  & \textbf{-8}  & \textbf{-8}  & \textbf{-8}  & \textbf{-8}  & \textbf{-8}  & \textbf{-8}  \\ \hline
		\multicolumn{1}{l}{sq4} & \textbf{-14}          & -13          & \textbf{-14} & \textbf{-14} & \textbf{-14} & \textbf{-14} & \textbf{-14} & \textbf{-14} \\ \hline
		\multicolumn{1}{l}{sq5} & \textbf{-23} & -22          & \textbf{-23} & -22          & -22          & \textbf{-23} & \textbf{-23} & \textbf{-23} \\ \hline
		\multicolumn{1}{l}{sq6} & \textbf{-21} & \textbf{-21} & \textbf{-21} & \textbf{-21} &              & \textbf{-21} & \textbf{-21} & \textbf{-21} \\ \hline
		\multicolumn{1}{l}{sq7} & -35          & -34          & -35          & -34          &              & -34          & \textbf{-36} & \textbf{-36} \\ \hline
		sq8                     & \textbf{-42} & -39          & \textbf{-42} & -37          &              & -32          & \textbf{-42} & -38          \\ \hline
	\end{tabular}
\end{table}

No caso das 3 primeiras sequências $sq1$, $sq2$ e $sq3$ a versão modificada dos AEMOs (M\_NSGAII e M\_IBEA) obtiveram os mesmos valores mínimos de energia que os estudos  anteriores considerados nesta comparação. No caso da sequência $sq4$, exceto pelo algoritmo M\_NSGAII, todos os outros atingiram o valor de -14. Observando a sequência $sq5$ 4 algoritmos mono-objetivo e o algoritmo M\_IBEA obtiveram o valor ótimo de 23. Entretanto, M\_NSGAII e os outros algoritmos obtiveram um valor inferior de -22.    

Já no caso da sequência $sq6$ todos os algoritmos obtiveram um valor ótimo de -21. Para a sequência $sq7$ o algoritmo M\_IBEA obteve -35 da mesma maneira que o EDA \cite{santana2008protein}. Entretanto o valor ótimo para sequência $sq7$ é -36 e foi obtido pelo NewACO \cite{shmygelska2003improved} e PERM \cite{hsu2003growth}. Para a última sequência $sq8$ o algoritmo M\_IBEA obteve o valor ótimo -42 mesmo valor que o EDA \cite{EDA} e NewACO \cite{shmygelska2003improved}. Todos as outras abordagens obtiveram valores inferiores para $sq8$. De maneira geral os resultados obtidos pelos AEMOs estão bem próximos dos resultados encontrados por outras abordagens em 6 de 8 sequências consideradas.


\subsection{Conclusão dos experimentos utilizando os AEMOs}

AEMOs são algoritmos evolucionários que visam a otimização de múltiplos objetivos ao mesmo tempo. Estes apresentam bons resultados quando aplicados a muitos problemas de várias áreas da ciência.  Neste experimentos dois AEMO foram aplicados ao problema PDP utilizando o modelo HP-2D. Duas abordagens multi-objetivas foram apresentadas: a primeira utiliza as versões padrão dos algoritmos IBEA e NSGAII; a segunda abordagem consistiu em modificar o IBEA e NSGAII adicionando a inicialização com \textit{backtrack} e o \textit{pool} de operadores, com objetivo de melhorar os resultados. Dado os resultados se tornou claro que as versões modificadas dos algoritmos tem uma maior habilidade em explorar o espaço de busca quando comparados com as versões padrão.  

Também foi possível verificar que a abordagem multi-objetiva utilizando o NSGAII e o M\_NSGAII também não apresentou resultados, tanto em termos de hypervolume como valores de energia, satisfatórios quando comparado com o IBEA e a versão modificada. Mesmo a versão padrão de IBEA obteve melhores resultados do que ambas as versões do NSGAII. Porém a versão modificada M\_IBEA apresentou os melhores resultados dentre os outros propostos nesta dissertação. Isto sugere que apenas uma formulação multi-objetiva não é suficiente para atingir bons resultados em termos de energia das estruturas. Com a inicialização via \textit{backtrack}, \textit{pool} de operadores e a maneira sofisticada de explorar o espaço de busca multi-objetivo do IBEA poderão possibilitar resultados promissores.

Dado os resultados é indiscutível que o algoritmo M\_IBEA foi capaz de escapar dos mínimos locais em quase todos os casos, exceto para $sq7$, por conta da capacidade, da modelagem multi-objetiva combinada com o \textit{pool} de operadores, em gerar diversidade entre a população. Também vale mencionar que os parâmetros para os algoritmos não foram \textit{tunados} e que existe uma chance de obterem melhores resultados caso um procedimento de \textit{tunning} seja feito.

Os resultados obtidos pelo M\_IBEA abrem um amplo campo de possibilidades para aprofundar formulações multi-objetivas para o problema PDP com o modelo HP-2D ou até mesmo o HP-3D. As descobertas desta dissertação motivam pesquisas adicionais utilizando abordagens multi-objetivas e adição de um \textit{pool} de operadores com objetivo de aprimorar a habilidade de escapar de mínimos locais. No caso de formulações multi-objetiva é possível mencionar que o \textit{design} de abordagens inovadoras como o uso de métricas para avaliar o grau de compactação das conformações de proteínas ou outros métodos que considerem diferentes fatores, podem aprimorar os AEMOs e possibilitar que melhores resultados sejam obtidos.

		% experimentação e validação
\chapter{Conclusão}


In this work  GEHyPSP, an automatic way of generating high level heuristics to a hyper-heuristic framework for the PSP problem, was presented and evaluated. The PSP is a very challenging problem  with a high number of local optima and a very complex landscape. Many authors explored the PSP problem with heuristic methods. 
However, very often the proposed heuristic approaches are unable to find the best known results when executing against longer sequences. Usually, the hyper-heuristic framework fits well in this kind of complex scenario. Hence, the goal of this paper was to generate, using a grammatical evolution strategy, selection mechanisms and acceptance criteria to a hyper-heuristic framework and evaluate its performance and behavior with a set of eleven HP instances. Three groups of experiments were executed, using three randomly selected HP instances, in order to generate the high level heuristics and later the best individuals found in the experiments were executed using all the eleven HP instances. 

Three groups of experiments were executed: first generating only selection mechanisms within a fixed acceptance criterion; second generating only acceptance criteria using the best selection mechanism found in the first; finally both high level heuristics were generated in parallel. The results showed that better high level heuristics were found when generating them separately. Unfortunately, when analyzing the behavior of the generated high level heuristics against the eleven instances it was possible to see that they were not able to achieve the best known results for the longer sequences. However, when comparing with a good state-of-art human-designed hyper-heuristic framework (GIHH) \cite{misir2012intelligent} the results are slightly close. This fact shows that it is possible to automate the creation of high level heuristics and obtain results close to the state-of-art hyper-heuristics frameworks. 

Another finding of this work was the behavior of the best generated acceptance criteria. It was noticed that it behaves just like "a better or equal" human-designed move acceptance strategy. Also some of the generated acceptance criteria were always accepting worst solution and this fact impacted in the individual fitness. This fact demonstrates that GEHyPSP was able to generate acceptance criteria with the same behavior of simple human-designed move acceptance strategies. However, in order to obtain better results it might be necessary to improve GEHyPSP to generate more complex selection mechanisms and acceptance criteria to couple with the landscape complexity of the PSP problem. 
		% conclusão

%=====================================================

% Estilos de bibliografia recomendados (só descomentar um estilo!)
\bibliographystyle{apalike-ptbr}	% [Maziero et al., 2006]
%\bibliographystyle{plain}		% [1], [1, 2]

% base de bibliografia (BibTeX)
\bibliography{refs}
%\bibliography{ref1, ref2, ref3} % se tiver mais de um arquivo BibTeX

%=====================================================

% apêndices
%\appendix
%\include{a1-exemplo/exemplo}

\end{document}

%=====================================================

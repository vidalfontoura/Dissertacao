\chapter{Introdução}

As proteínas são responsáveis por muitas funções importantes das células vivas. Estas estruturas garantem o correto funcionamento de um amplo número de entidade biológicas. As proteínas são o produto de um processo chamado de dobramento de proteínas, no qual uma cadeia de aminoácidos inicialmente desdobrada será transformada em sua estrutura final/nativa. A predição de estruturas de proteínas possui um campo amplo de aplicações biotecnológicas e médicas. Por exemplo: síntese de novas proteínas e dobramentos \cite{wang2012structural, rothlisberger2008kemp}, síntese de novas drogas baseada nas estruturas \cite{qian2004improvement, krieger2009improving}  e obtenção experimental de estruturas a partir de dados incompletos de ressonância magnética nuclear \cite{shen2009novo}.  

Determinar a estrutura nativa de proteínas uma tarefa desafiadora até mesmo para modernos super computadores. Isto ocorre por conta do imenso espaço de busca para testar todas as possíveis configurações que uma dada proteína pode adotar. Diferentes formas de representar as estruturas/conformações de proteínas existem e podem ser utilizadas para simular o processo de dobramento. Embora existam modelos extremamente detalhados, estas representações são computacionalmente muito custosas. Consequentemente, muitos autores \cite{custodio2004investigation,hsu2003growth,lin2011protein,unger1993genetic,santanna2008,custodio2014multiple, garza2012locality} utilizam modelos simplificados para representar as estruturas de proteínas. Um modelo bastante conhecido para este propósito, criado por Lau and Dill \cite{lau1989lattice}, é o modelo Hidrofóbico-Polar (HP). Este modelo simplifica os amino ácidos em apenas dois tipos: hidrofóbico (H) e polar (P). 



Também existe uma grande variação das características do espaço de busca entre diferentes instâncias do modelo HP. \textit{Grids} bidimensionais (2D) e tridimensionais (3D) pode ser utilizados para representar as conformações que uma proteína pode adotar. Cada conformação no \textit{grid} tem um valor de energia associado.

Muitos estudos  \cite{custodio2004investigation,hsu2003growth,lin2011protein,unger1993genetic,santana2008component,custodio2014multiple, garza2012locality} have used the Hydrophobic-Polar (HP) \cite{lau1989lattice}
 partir do processo de dobramento. Para avaliar as estruturas representadas pelo modelo HP é preciso computar o valor de energia associado a uma dada conformação \cite{unger1993genetic}. Para isto, é necessário considerar as interações entre os aminoácidos. Uma interação ocorre quando o par de aminoácidos é adjacente no \textit{grid}/cubo e não é adjacente na sequência. No modelo HP existem apenas três: (HP, HH e PP), mas somente interações hidrofóbicas (HH) influenciam no valor de energia referente a uma conformação \cite{unger1993genetic}. A questão que surge é como buscar, dentre as possíveis conformações, aquela cuja a energia seja mínima.
